
\section{Принцип максимума и минимума для гармонических функций.}
%Никитушка отмочил красоту
\begin{theorem}
{\bf(Принцип максимума)} Если $u(x)$ -- гармоническая в
области $\Omega$ и достигает $\max$ или $\min$ значения в
точке $a \in \Omega$, то $u(x) \equiv u(a) \Forall x \in \Omega$.

\begin{offtop}
Теорема справедлива в $\R^n$.
\end{offtop}

\begin{proof}

\begin{itemize}
\item
\item 
Докажем вспомогательное локальное 
\begin{statement}
\label{statement_22.1}
Пусть $u(x) \in C^2(\Omega)$ достигает максимума в точке $a$, а так же удовлетворяет {\bf свойству среднего}:

\[
u(a) = \dfrac{1}{4\pi r^2}\oint\limits_{\abs*{y-a} = r}u(y
)\;dS_y \Forall r: 0 < r < d_a = \dist{(a, \R^3 \backslash
 \Omega)}.
\]

Тогда $u(x) \equiv u(a) \Forall x \in B(a, d_a)$ .

\end{statement}
\begin{proof}
$
u(a) = \dfrac{1}{4\pi r^2}\oint\limits_{\abs*{y-a} = r}u(y
)\;dS_y  =
\dfrac{u(a)}{4\pi r^2}\oint\limits_{\abs*{y-a} = r} \;dS_y  +
 		\dfrac{1}{4\pi r^2}\oint\limits_{\abs*{y-a} = r}\brk*{u(y
)-u(a)}\;dS_y =
	u(a) + \dfrac{1}{4\pi r^2}\oint\limits_{\abs*{y-a} = r}\brk*{u(y
)-u(a)}\;dS_y \Rightarrow 
	\oint\limits_{\abs*{y-a} = r}\underbrace{\brk*{u(y)-u(a)}}_{\le 0,\ \text{непрерывна}}\;dS_y = 0
	 \Leftrightarrow
	\quad u(y) = u(a) \Forall y: |y-a|=r < d_a.
 $
\end{proof}

\item
{\bf Докажем саму теорему}
\begin{center}
\includegraphics[scale=0.5]{22_1_new}
\end{center}
Соединим $a$ и $b$ кусочно-гладкой кривой. Эту кривую параметризуем натуральным параметром (параметризация кривой длиной её дуги): 
$x = x(s), \; x(0) = a, \; x(L) = b$.
{\bf Обозначим}
$d = \dist{\brk[c]*{\gamma = x; \partial{\Omega}}} > 0$
($d$ действительно $> 0$ : если $d = 0$, то 
$\exists\{x_n\}_{n=1}^\infty \subset \gamma: \rho\brk*{x_n, \partial\Omega}\to 0$.
 Выделим из $\{x_n\}$ сходящуюся $\{x_{n_k}\}=\{y_k\}$($\gamma$ - ограничено). Пусть $y_k \to y_0$. Тогда $y_0$ - предельная для
 $\partial \Omega$ в силу замкнутости 
 $y_0 \in \gamma \cup \partial \Omega \Rightarrow$ противоречие).\\
 Разобьем $[0,L]$ на части размера $\Delta S = \dfrac{L}{N}$.\\
 Пусть $\brk[c]*{S_k = k\Delta S, \; x_k = x(S_k)}$. 
 Число N выберем так, чтобы $\Delta S = \dfrac{L}{N} < d$\\
 \begin{center}
 \includegraphics[scale=0.5]{22_2_new}
 \end{center}
 Заметим, что $\abs*{x_k - x_{k-1}} \boxed{\leq} S_k - S_{k-1} < d$ ($\boxed{\cdot}$ - кратчайшее расстояние между точками это отрезок).
 \begin{center}
 \includegraphics[scale=0.5]{22_3_new}
\end{center}
 Рассмотрим шары $\{B_d(x_k)\}_{k=0}^N$. В силу
 $\abs*{x_k - x_{k-1}}< d$ верно $x_{k+1} \in B_d(x_k)$. 
  Примем утверждение (\ref{statement_22.1}) к первому шару. 
  Как следствие $u(x_1) = u(a)\Rightarrow$ утверждение (\ref{statement_22.1}) применимо уже ко второму шару.
   В цепочке шаров конечное число 
  $\Rightarrow$ добираемся до точки $b$ - теорема доказана.  
\end{itemize}
\end{proof}
\end{theorem}

$\bullet$ Заметим, что достаточно было потребовать
свойство среднего и непрерывность, вместо гармоничности.


\begin{conseq}
\label{conseq22.1}
Пусть $\Omega$ -- ограниченная область, а $u(x)$ -- гармоническая в $\Omega$ и непрерывная на $\overline{\Omega}$.
Тогда $u(x)$ достигает $\max$ и $\min$ на $\partial \Omega$, т.е.
$\min\limits_{y \in \partial \Omega}u(y) \leq u(x) \leq \max\limits_{y \in \partial \Omega}u(y)$
	\begin{proof}
	Либо максимум/минимум на границе, либо $u(x)\equiv const$ в $\Omega$.
	\end{proof}
\end{conseq}


\begin{conseq}
Для указанной в следствии (\ref{conseq22.1}) $u(x) \hookrightarrow \abs*{u(x)} \leq \max\limits_{y \in \partial \Omega} \abs*{u(y)}$.
\begin{proof}

\begin{equation}
  \begin{cases}
  u(x) \leq \max\limits_{\partial \Omega}u(y) \leq \max\limits_{\partial \Omega}\abs*{u(y)}\\  
  -u(x) \leq \max\limits_{\partial \Omega}(-u(y)) \leq \max\limits_{\partial \Omega}\abs*{u(y)}\\
  \end{cases}
  \end{equation}  
  
  $\Rightarrow \abs*{u(x)} \leq \max\limits_{\partial \Omega}\abs*{u(y)}$.
\end{proof}
\end{conseq}

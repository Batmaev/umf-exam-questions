\section{Решение методом Фурье смешанной задачи для уравнения теплопроводности на отрезке с однородными краевыми условиями Дирихле. Существование и единственность классического решения.}
% Техал Витяй
Рассмотрим смешанную (начально-краевую) задачу:
\begin{equation} \label{15_1}
	\begin{cases}
		u_t - a^2u_{xx} = f(t,x), \quad &0 < t < T, \quad 0 < x < l, \\
		\eval{u}_{t=0} = u_0(x), \quad & 0 \leqslant x \leqslant l, \\
		\eval{u}_{x=0} = \psi_0(t), \quad \eval{u}_{x=l} = \psi_1(t), \quad & 0 \leqslant t \leqslant T.
	\end{cases}
\end{equation}
Рассматриваем ее классическое решение~-- функцию $u(t,x) \in C^{1,2}_{t,x}(Q_T) \cap C(\overline{Q_T})$, где \\ $Q_T = \brk[c]*{(t,x) : t \in (0,T), \, x \in (0,l)}$, удовлетворяющую в $Q_T$ уравнению, начальному и граничным условиям.
\begin{theorem}[Единственности]
	Не может существовать более одного классического решения задачи (\ref{15_1}).
\end{theorem}
\begin{proof}
	Если $u_1, u_2$~-- классические решения, то $v = u_1 - u_2$~-- классическое решение полностью однородной задачи. На параболической границе $\Gamma_T = \brk[c]*{t=0, x \in \brk[s]*{0,l}} \cup \brk[c]*{x=0, t \in \brk[s]*{0,T}} \cup \brk[c]*{x=l, t \in \brk[s]*{0,T}}$ $\eval{v}_{\Gamma_T} = 0$. Но на $\Gamma_T$ достигается максимум и минимум $v$ в $Q_T$ в силу принципа максимума $\Rightarrow v \equiv 0$
\end{proof}

Частный случай, указанный в билете:
\begin{equation*}
	\begin{cases}
		u_t - a^2u_{xx} = 0, \quad & 0 < t < T, \quad 0 < x < l, \\
		\eval{u}_{t=0} = u_0(x), \quad & 0 \leq x \leq l, \\
		\eval{u}_{x=0} = 0, \quad \eval{u}_{x=l} = 0, \quad & 0 \leqslant t \leqslant T.
	\end{cases}
\end{equation*}
Из непрерывности естественно требовать выполнение условий согласования: $u_0(0) = u_0(l) = 0$. Оказывается, в таких условиях решение существует.

{\it Метод Фурье}~-- поиск решения в виде ряда по собственным функциям стационарного оператора.

Придём к этой идее. Будем искать решение $Lu = u_t - a^2u_{xx} = 0$ методом разделения переменных: \\ $u(t,x) = \Theta(t) X(x), \quad u(t,x) \not\equiv 0$.

Подставляем: $\dot{\Theta}(t) X(x) - a^2\Theta(t) X''(x) = 0 \Rightarrow \dfrac{\dot{\Theta}(t)}{a^2\Theta(t)} = \dfrac{X''(x)}{X(x)} = -\lambda = \text{const}$, т.к. равенство выполнено $\forall (t,x) \in Q_T$

Получаем на функции $\Theta$ и $X$ следующие уравнения:
\begin{equation*}
	\begin{cases}
		-X''(x) = \lambda X(x), \quad & 0 \leq x \leq l, \\
		\dot{\Theta}(t) + \lambda a^2 \Theta(t) = 0, \quad & 0 \leq t \leq T.
	\end{cases}
\end{equation*}
Из начального условия $u(t,0) = \Theta(t)X(0) \Forall t \in (0,T) \Rightarrow X(0) = 0$. Аналогично $X(l) = 0$.

Задача для $X$:
\begin{equation}
	\begin{cases}
		-X''(x) = \lambda X(x), \quad x \in (0,l), \\
		X(0) = X(l) = 0, \\
		X(x) \not\equiv 0.
	\end{cases}
\end{equation}
Поставленная задача называется {\it задачей Штурма-Лиувилля}.

Введем оператор $A$:
\begin{itemize}
	\item $\mathrm{D}(A) = \brk[c]*{X \in C^2[0,l] : X(0) = X(l) = 0}$
	\item $\mathrm{Im}(A) = \brk[c]*{Y \in C[0,l]}$
	\item $AX = -\Delta X = Y$
\end{itemize}

Задача Штурма-Лиувилля~-- это задача на собственные функции и собственные значения оператора $A$.

Решим ее:
\begin{itemize}
	\item $\lambda < 0$:

		$X(x) = C_1e^{\sqrt{\abs*{\lambda}}x} + C_2e^{-\sqrt{\abs*{\lambda}}x}$

			$\begin{cases}
				X(0) = C_1 + C_2 = 0, \\
				X(l) = C_1e^{\sqrt{\abs*{\lambda}}l} + C_2e^{-\sqrt{\abs*{\lambda}}l} = 0.
			\end{cases}$

			$\text{det}\begin{pmatrix} 1 & 1 \\ e^{2\sqrt{\abs*{\lambda}}l} & 1 \end{pmatrix} = 1 - e^{2\sqrt{\abs*{\lambda}}l} = 0 \Rightarrow \sqrt{\abs*{\lambda}}l = 0$, противоречие.

			Итак, <<$-\Delta$>> с граничными условиями Дирихле не имеет отрицательных собственных значений.

	\item $\lambda = 0$:

		$X(x) = C_1x + C_2$

		$\begin{cases}
			X(0) = C_2 = 0, \\
			X(l) = C_1l = 0.
		\end{cases}$

		Нетривиальных решений нет.

	\item $\lambda > 0$:

		$X(x) = C_1\cos\sqrt{\lambda}x + C_2\sin\sqrt{\lambda}x$

		$\begin{cases}
			X(0) = C_1 = 0, \\
			X(l) = C_2\sin\sqrt{\lambda}l = 0.
		\end{cases}$

		$\sqrt{\lambda}l = \pi k \Rightarrow \lambda_k = \brk*{\dfrac{\pi k}{l}}^2, k \in \N$

		Функции $X_k(x) = \sin\brk*{\dfrac{\pi k}{l}x}$
\end{itemize}
Теперь для найденных $\lambda_k$ решаем $\dot{\Theta}_k(t) + \lambda_k a^2 \Theta_k(t) = 0 \Rightarrow \Theta_k(t) = e^{-a^2\lambda_kt}$

Мы нашли $u_k(t,x) = e^{-a^2\lambda_kt}\sin\brk*{\dfrac{\pi k}{l}}$~-- счетное число бесконечно гладких решений: $u_k$ удовлетворяет задаче

\begin{equation*}
	\begin{cases}
		\brk*{u_k}_t - a^2\brk*{u_k}_{xx} = 0, \\
		u_k(t,0) = u_k(t,l) = 0, \\
		u_k(0,x) = X_k(x) = \sin\lambda_kx.
	\end{cases}
\end{equation*}
Тогда $u_A(t,x) = \displaystyle\sum\limits_{k=1}^N A_k u_k(t,x)$~-- решение для задачи с начальным условием $u(0,x) = \displaystyle\sum\limits_{k=1}^N A_k X_k(x)$.

Обозначим $Ku_0$~-- класс функций $u_0 = \displaystyle\sum\limits_{k=1}^N A_k X_k(x)$~-- тех, для которых умеем выписать явное решение. Пусть $A_k$~-- $k$-мерные векторы. Между $Ku_0$ и $A_k$ есть биекция (по $u_0 = \displaystyle\sum\limits_{k=1}^N A_k X_k(x)$ однозначно восстанавливаем $A_k = \dfrac{2}{l}\displaystyle\int\limits_0^l u_0(x) \sin\brk*{\dfrac{\pi k}{l}x}dx$)

Бесконечномерный вектор подойдет уже не всегда. Как минимум ряд $\displaystyle\sum\limits_{k=1}^\infty A_k X_k(x)$ должен сойтись в замыкании области. Функция $\displaystyle\sum\limits_{k=1}^\infty A_k u_k(t,x)$ должна быть нужной гладкости, а также удовлетворять уравнению (\ref{15_1}).

\begin{statement}
	$\brk[c]*{A_k}_{k=1}^{\infty}: \displaystyle\sum\limits_{k=1}^{\infty} \abs*{A_k} < \infty$ подойдет.
\end{statement}

\begin{proof}
	Пусть $u_0(x) = \displaystyle\sum\limits_{k=1}^{\infty} A_k\sin\brk*{\dfrac{\pi k}{l}x}, \displaystyle\sum\limits_{k=1}^{\infty} \abs*{A_k} < \infty$. Тогда $\abs*{A_k\sin\brk*{\dfrac{\pi k}{l}x}} \leqslant \abs*{A_k} \Rightarrow$ по теореме Вейерштрасса ряд сходится абсолютно и равномерно $\Rightarrow$ сумма непрерывна.

	Равномерно сходящийся ряд можно почленно интегрировать $\Rightarrow$ по $u_0(x)$ восстанавливаем $A_n$: $\displaystyle\int\limits_0^l u_0(x)\sin\brk*{\dfrac{\pi n}{l}x}dx = \displaystyle\sum_{k=1}^{\infty} A_k \displaystyle\int\limits_0^l \sin\brk*{\dfrac{\pi k}{l}x} \sin\brk*{\dfrac{\pi n}{l}x} dx \Rightarrow A_n = \dfrac{2}{l} \displaystyle\int\limits_0^l u_0(x) \sin\brk*{\dfrac{\pi n}{l}x} dx$

	Теперь рассмотрим ряд $u_A(t,x) \sim \displaystyle\sum\limits_{k=1}^{\infty} A_k e^{-\brk*{\frac{a\pi k}{l}}^2t} \sin\brk*{\dfrac{\pi k}{l}x}$

	Пока не можем поставить знак равенства, поскольку еще не выяснили сходимость.

	$\abs*{A_k e^{-\brk*{\frac{a\pi k}{l}}^2t} \sin\brk*{\dfrac{\pi k}{l}x}} \leqslant \abs*{A_k} \Rightarrow$ ряд сходится абсолютно и равномерно, и мы можем поставить знак равенства:

	$u_A(t,x) = \displaystyle\sum\limits_{k=1}^{\infty} A_k e^{-\brk*{\frac{a\pi k}{l}}^2t} \sin\brk*{\dfrac{\pi k}{l}x}$

	Покажем, что получилась $u_A(t,x) \in C^\infty(t>0, \, 0 \leqslant x \leqslant l)$. 

	Возьмем прямоугольник $Q_\delta = \brk[c]*{(t,x) : t \geqslant \delta > 0, \, 0 \leqslant x \leqslant l}$. 

	Формально $\dfrac{\partial u_A}{\partial t} \sim -\displaystyle\sum\limits_{k=1}^{\infty} A_k\brk*{\dfrac{a\pi k}{l}}^2 e^{-\brk*{\frac{a\pi k}{l}}^2t} \sin\brk*{\dfrac{\pi k}{l}x} = -\displaystyle\sum\limits_{k=1}^{\infty} \varphi_k(t,x)$

	Для краткости введём $y = \brk*{\dfrac{a\pi k}{l}}^2$.

	Оценка: $\abs*{\varphi_k} \leqslant \abs*{A_k}ye^{-y\delta} = \abs*{A_k} \dfrac{1}{\delta} (y\delta) e^{-y\delta} \leqslant \abs*{A_k} \dfrac{1}{\delta e}$

	Последнее неравенство следует из того, что функция $xe^{-x}$ имеет максимум в точке $x = 1$, равный $\dfrac{1}{e}$.

	Итак, по теореме Вейерштрасса, ряд сходится абсолютно и равномерно.

	Варьируя $\delta$, прямоугольниками $Q_\delta$ заметаем всю область $\brk[c]*{t > 0, 0 \leqslant x \leqslant l}$

	Для остальных производных получим то же самое~-- всегда будет получаться произведение многочлена на экспоненту с отрицательным показателем.

	Мы научились решать задачу для $u_0(x) = \displaystyle\sum\limits_{k=1}^{\infty} A_k \sin\brk*{\dfrac{\pi k}{l}x}$.
\end{proof}

Докажем серию лемм.
\begin{lemma}
	Пусть в гильбертовом пространстве $\mathcal{H}$ оператор $A$ симметричный (самосопряжённый), т.е. $(Ax,y) = (x,Ay)$. Тогда:
	\begin{enumerate}
		\item Все собственные значения $A$ вещественны;
		\item Собственные функции, отвечающие различным собственным значениям, ортогональны.
	\end{enumerate}
\end{lemma}
\begin{proof}
	Пусть $x_k$~-- собственный вектор, отвечающий собственному значению $\lambda_k$, а $x_n$~-- собственный вектор, отвечающий собственному значению $\lambda_n$, причем $\lambda_k \not= \lambda_n$. Тогда:
	\begin{enumerate}
		\item $\lambda_k(x_k, x_k) = (Ax_k, x_k) = (x_k, Ax_k) = (x_k, \lambda_kx_k) = \overline{\lambda_k}(x_k, x_k) \Rightarrow \lambda_k = \overline{\lambda_k} \Rightarrow \mathrm{Im} \, \lambda_k = 0$
		\item $\lambda_k(x_k, x_n) = (Ax_k, x_n) = (x_k, Ax_n) = \overline{\lambda_n}(x_k,x_n) \underset{\text{пункт 1}}{=} \lambda_n(x_k,x_n) \Rightarrow \underset{\not= 0}{\underline{(\lambda_k - \lambda_n)}}(x_k,x_n) = 0 \Rightarrow (x_k,x_n) = 0$
	\end{enumerate}
\end{proof}
\begin{lemma}
	Оператор $A = -\dfrac{d^2}{dx^2}$, определенный на $\mathrm{D}(A)$, является симметричным относительно скалярного произведения в $L_2[0,l]: (u,v) = \displaystyle\int\limits_0^l u(x) \overline{v(x)} dx$.
\end{lemma}
\begin{proof}
	$(Au,v) = \displaystyle\int\limits_0^l (-u''(x))\overline{v(x)}dx = \underset{= 0, \text{ в силу определения $\mathrm{D}(A)$}}{\underline{\left.-u'(x)\overline{v(x)}\right|_0^l}} + \displaystyle\int\limits_0^l u'(x) \overline{v'(x)} dx = \underset{= 0, \text{ в силу определения $\mathrm{D}(A)$}}{\underline{\left.u(x)\overline{v'(x)}\right|_0^l}} + \displaystyle\int\limits_0^l u(x)\brk*{-\overline{v''(x)}}dx = (u, Av)$
\end{proof}
\begin{lemma}
	Пусть $\brk[c]*{e_k}$~-- не более чем счетная ортогональная система в линейном пространстве со скалярным произведением: $(e_k, e_j) = \delta_{kj} \underset{>0}{\underline{(e_k, e_k)}}$. Тогда $\forall f$ из этого пространства справедливо неравенство Бесселя: $\displaystyle\sum\limits_{k=1}^{\infty} \abs*{c_k}^2 (e_k, e_k) = \displaystyle\sum\limits_{k=1}^{\infty} \abs*{\dfrac{(f,e_k)}{(e_k,e_k)}}^2 (e_k, e_k) \leq (f, f)$.
\end{lemma}
\begin{proof}
	$0 \leq \brk*{f - \displaystyle\sum\limits_{k=1}^n c_ke_k, f - \displaystyle\sum\limits_{k=1}^n c_ke_k} = (f,f) - \displaystyle\sum\limits_{k=1}^n c_k(e_k,f) - \displaystyle\sum\limits_{j=1}^n \overline{c_j} (f,e_j) + \\ + \displaystyle\sum\limits_{k=1}^n \displaystyle\sum\limits_{j=1}^n c_k \overline{c_j} (e_k, e_j) = (f,f) - \displaystyle\sum\limits_{k=1}^n c_k \overline{c_k} (e_k, e_k) - \displaystyle\sum\limits_{j=1}^n c_j \overline{c_j} (e_j, e_j) + \displaystyle\sum\limits_{i=1}^n c_i \overline{c_i} (e_i, e_i) = \\ = (f,f) - \displaystyle\sum\limits_{k=1}^n \abs*{c_k}^2 (e_k, e_k)$. 
    
    Переходя к пределу при $n \to \infty$, получаем $\displaystyle\sum\limits_{k=1}^{\infty} \abs*{c_k}^2 (e_k, e_k) \leq (f,f)$.
\end{proof}
\begin{lemma}
	Пусть два ряда $\displaystyle\sum\limits_{k=1}^{\infty} \abs*{\alpha_k}^2 = A, \: \displaystyle\sum\limits_{k=1}^{\infty} \abs*{\beta_k}^2 = B$ сходятся. Тогда ряд $\displaystyle\sum\limits_{k=1}^{\infty} \alpha_k \beta_k$ сходится абсолютно, причем $\displaystyle\sum\limits_{k=1}^{\infty} \abs*{\alpha_k \beta_k} \leq \sqrt{A} \sqrt{B}$.
\end{lemma}
\begin{proof}
	$\displaystyle\sum\limits_{k=1}^{n} \abs*{\alpha_k \beta_k} \underset{\text{КБШ}}{\leq} \sqrt{\displaystyle\sum\limits_{k=1}^{n} \abs*{\alpha_k}^2} \sqrt{\displaystyle\sum\limits_{k=1}^{n} \abs*{\beta_k}^2}$. Переходя к пределу при $n \to \infty$, получаем требуемое.
\end{proof}
\begin{lemma}
	Пусть $v(x) \in C^1[0,l], \, v(0) = v(l) = 0$. 

	Тогда ряд $\displaystyle\sum\limits_{k=1}^{\infty} A_k \sin\brk*{\dfrac{\pi k}{l}x}$, где $A_k = \dfrac{2}{l}\displaystyle\int\limits_0^l v(y)\sin\brk*{\dfrac{\pi k}{l}y}dy$, сходится на $[0,l]$ к $v(x)$ абсолютно и равномерно.
\end{lemma}
\begin{proof}
	\begin{enumerate}
		\item Система $\brk[c]*{e_k} = \brk[c]*{\sin\brk*{\dfrac{\pi k}{l}x}}$ ортогональна относительно скалярного произведения в $L_2 [0,l]$, так как состоит из собственных функций оператора <<$-\Delta$>> с однородными условиями Дирихле~-- симметричного в $L_2 [0,l]$ оператора.
		\item $A_k = \dfrac{(v,e_k)}{(e_k,e_k)},$ ряд $\displaystyle\sum\limits_{k=1}^{\infty} \abs*{A_k}^2 < \infty$ по неравенству Бесселя.
		\item $A_k = \underset{=0}{\underline{-\left.\dfrac{2}{l} \dfrac{l}{\pi k} v(y) \cos\brk*{\dfrac{\pi k}{l}y}\right|_0^l}} + \dfrac{2}{l} \dfrac{l}{\pi k} \displaystyle\int\limits_0^l v'(y) \cos\brk*{\dfrac{\pi k}{l}y}dy = \dfrac{l}{\pi k} \alpha_k$, где $\alpha_k = \dfrac{2}{l} \displaystyle\int\limits_0^l v'(y) \cos\brk*{\dfrac{\pi k}{l}y}dy$.
		\item Ряд $\displaystyle\sum\limits_{k=1}^{\infty} \abs*{\alpha_k}^2 < \infty$ по неравенству Бесселя, т.к. система $\brk[c]*{g_k} = \brk[c]*{\cos\brk*{\dfrac{\pi k}{l}x}}$ ортогональна относительно скалярного произведения в $L_2 [0,l]$, так как состоит из собственных функций оператора <<$-\Delta$>> с однородными условиями Неймана~-- симметричного в $L_2 [0,l]$ оператора.

		Ряд $\displaystyle\sum\limits_{k=1}^{\infty} \dfrac{1}{k^2} = \dfrac{\pi^2}{6}$ сходится. Тогда сходится абсолютно ряд $\displaystyle\sum\limits_{k=1}^{\infty} \dfrac{\alpha_k}{k} \Rightarrow \displaystyle\sum\limits_{k=1}^{\infty} \abs*{A_k} < \infty$. 

		Функция $\varphi(x) = \displaystyle\sum\limits_{k=1}^{n} A_k \sin\brk*{\dfrac{\pi k x}{l}}$ непрерывна.
		\item Сходимость к $v(x)$: Построим
			\begin{equation*}
				\tilde{v}(x) = \begin{cases} v(x), & x \in [0,l] \\ -v(-x), & x \in [-l,0] \end{cases}
			\end{equation*}
			Затем продолжим на $\R$, сделав периодической: $\tilde{v}(x+2l) = \tilde{v}(x)$.
			Получаем непрерывную периодическую функцию, а во всех точках $x \in [0,l] \Exists \tilde{v}'_-(x), \tilde{v}'_+(x)$. Тогда ряд Фурье этой функции сходится к ней на всей $\R$. В силу нечетности $\tilde{v}$, этот ряд~-- только по синусам, а коэффициенты Фурье равны $A_k$ (для них совпадают формулы). Значит, на $[0,l]$ имеем $v(x) = \displaystyle\sum\limits_{k=1}^{\infty} A_k \sin\brk*{\dfrac{\pi k}{l}x}$
	\end{enumerate}
\end{proof}
Таким образом, доказана теорема:
\begin{theorem}
	Пусть в смешанной задаче
	\begin{equation*}
		\begin{cases}
			u_t - a^2u_{xx} = 0, \quad & 0 < t < T, \quad 0 < x < l, \\
			\eval{u}_{t=0} = u_0(x), \quad & 0 \leq x \leq l, \\
			\eval{u}_{x=0} = \eval{u}_{x=l} = 0, \quad & 0 \leq t \leq T.
		\end{cases}
	\end{equation*}
	функция $u_0(x)$ удовлетворяет условиям гладкости $\brk*{u_0 \in C^1[0,l]}$ и согласования $\brk*{u_0(0) = u_0(l) = 0}$. Тогда ряд $\displaystyle\sum\limits_{k=1}^{\infty} A_k e^{-\brk*{\frac{a\pi k}{l}}^2t} \sin\brk*{\dfrac{\pi k}{l}x} = u(t,x)$, где $A_k = \dfrac{2}{l} \displaystyle\int\limits_0^l u_0(x) \sin\brk*{\dfrac{\pi k}{l}x} dx$, сходится абсолютно и равномерно в $\overline{Q_T} = [0,T] \times [0,l]$, функция $u(t,x) \in C(\overline{Q_T}) \cap C^{\infty}(Q_T)$ и является классическим решением этой задачи, а любая производная при $t > 0$ от $u(t,x)$ может быть найдена почленным дифференцированием.
\end{theorem}

\documentclass[../main.tex]{subfiles}
\begin{document}

\section[Задача Коши в \texorpdfstring{$\R^n$}{R\textasciicircum n}]{Постановка задачи Коши для дифференциального уравнения 2-го порядка с частными производными с линейной старшей частью в $\R^n$. Понятие о характеристической поверхности.}
% Затехал: Рязанов Андрей

\imaginarySubsection{Постановка}
В области $\Omega \subset \R^n$ задано уравнение:
\begin{equation}
\label{eq::2::init}
\sum\limits_{i,j=1}^{n} a_{ij}(x)\frac{\partial^2u}{\partial x_i\partial x_j} + F(x, u, \nabla u) = 0
\end{equation}
и поверхность $S\colon\ \omega(x) = \omega(x_1, \dots, x_n) = 0,\ \omega \in C^2(\Omega)$ и $\grad{\omega} \not= 0$ на $\Omega$ (нет особых точек).\\
На поверхности задано гладкое некасательное поле $\vec{\nu} = (\nu_1(x), \dots, \nu_n(x)),\ \; \langle \vec{\nu}, \vec{n} \rangle \not= 0$.
\begin{definition}[Задача Коши]
В $U(x^0)\subset\Omega,\ x^0 \in S$, найти то решение уравнения \eqref{eq::2::init}, которое удовлетворяет двум условиям:
\begin{enumerate}
\item $u(x)|_{S\cap U(x^0)}=u_0(x)$
\item $\eval*{\pd{u}{\vec{\nu}}}_{S\cap U(x^0)} = u_1(x)$ -- выводящая производная

Здесь введена производная по направлению: $\pd{u}{\vec{\nu}} = \brk*{\vec{\nu},\, \nabla u} = \displaystyle\sum\limits_{k=1}^{n} \nu_k(x)\, \dfrac{\partial u}{\partial x_k}(x) $
\end{enumerate}
\end{definition}
Может не быть непрерывной зависимости от начальных данных.\\
Функции $u_0, u_1$ произвольно брать, вообще говоря, нельзя.


Далее определим характеристическую поверхность.\\
Пусть $S$ -- гиперплоскость $x_n = 0$. Нормаль $\vec{n} = (0,0,\dots,0, 1)^\Transp,\\
u(x_1, \dots, x_{n-1}, 0) = u_0(x_1, \dots, x_{n-1}),\quad \dfrac{\partial u}{\partial\vec{n}} = \dfrac{\partial u}{\partial x_n} = u_1(x_1, \dots, x_{n-1}) $

Мы знаем значения функции на гиперплоскости.\\
Знаем градиент:
\begin{equation*}
\left\{\begin{aligned}
\frac{\partial u}{\partial x_1}(x_1, \dots, x_{n-1}, 0) &= \frac{\partial u_0}{\partial x_1}(x_1, \dots, x_{n-1})\\ 
&\vdots \\
\frac{\partial u}{\partial x_{n-1}}(x_1, \dots, x_{n-1}, 0) &= \frac{\partial u_0}{\partial x_{n-1}}(x_1, \dots, x_{n-1})
\end{aligned} \right\}
+\left\{\frac{\partial u}{\partial x_n}(x_1,\dots,x_{n-1},0) = u_1 \right\}
\end{equation*}
Мы знаем и вторые производные: берём указанные сверху производные и дифференцируем вдоль поверхности, получим \[\frac{\partial^2u}{\partial x_i\partial x_j}(x_1,\dots,x_{n-1}, 0) = \frac{\partial^2u_0}{\partial x_i\partial x_j},\ 1\le i,j \le n-1 \]
\[\frac{\partial^2u}{\partial x_n\partial x_i}(x_1,\dots,x_{n-1}, 0) = \frac{\partial u_1}{\partial x_i}(x_1,\dots,x_{n-1}),\ 1\le i\le n-1\]

Не нашли только $\pd{u}{x_n}{x_n}$. До этого мы вообще еще не использовали уравнение:

$$\sum\limits_{i,j=1}^{n-1}a_{ij}u_{x_ix_j} + \sum\limits_{j=1}^{n-1}\left[a_{nj}u_{x_nx_j} + a_{jn}u_{x_jx_n}\right] + \underbrace{a_{nn}u_{x_nx_n}}_{\substack{\text{только это слагаемое}\\ \text{еще не определено}}} + F(x, u, \nabla u) = 0 $$
Если $a_{nn}(x) = 0$ на нашей гиперплоскости, то эту гиперплоскость назовём {\bf характеристической}. На характеристической гиперплоскости полученное уравнение задаёт функциональную связь $u_0$ и $u_1$ -- эта связь называется {\bf условием совместности}.

Теперь переходим к произвольной поверхности: заменим координаты так, чтобы локально поверхность была гиперплоскостью:
\[y = y(x) = \begin{cases} y_1 = y_1(x_1,\dots, x_n) \\ \vdots \\ y_{n-1} = y_{n-1}(x_1,\dots,x_n) \\ y_n = \omega(x_1,\dots,x_n) \end{cases} \]
$\Rightarrow$ после преобразования $\omega=0 \Leftrightarrow y_n=0$.\\
Найдём это преобразование: возьмем $\vec{n} = \nabla \omega(x_0)$.
\vspace{0.2em}
\begin{center}
\includegraphics[width=0.34\textwidth]{./pic 3_1.pdf}
\end{center}
\vspace{-0.7em} % Тень на картинке простирается далеко вниз, поэтому кажется, что там есть пустое место. Чтобы его убрать, делаю отрицательное vspace
Дополним $\vec{n}$ до базиса -- получим $\brk[a]*{\vec{l}_1,\dots,\vec{l}_{n-1},\vec{n}}$. Ортогонализуем -- получим $\brk[a]*{\vec{e}_1,\dots,\vec{e}_{n-1},\vec{n}}.$ Возьмем такое преобразование:
\[ \vec{y}(\vec{x}) = \begin{cases} y_1 = \brk*{\vec{x}-\vec{x}_0} \cdot \vec{e}_1 \\ \vdots \\ y_{n-1} = \brk*{\vec{x}-\vec{x}_0} \cdot \vec{e}_{n-1} \\ y_n = \omega(\vec{x}) \end{cases}\]
Проверим, что якобиан не равен 0:
$$
J = \begin{vmatrix} 
\pd{y_1}{x_1} & \cdots & \pd{y_1}{x_n} \\
\cdots &\cdots &\cdots \\
\pd{\omega}{x_1} & \cdots & \pd{\omega}{x_n} \end{vmatrix}
= \begin{vmatrix}
\;(\qquad e_1^\transp \qquad)\;\\[0.2em]
\cdots\\[0.2em]
\;(\qquad e_n^\transp \qquad)\;\\[0.6em]
\;(\;\;\: \nabla\omega(x)^\Transp \;\;\:)\;
\end{vmatrix}
\overset{\text{$x = x_0$}}{=\joinrel=}
\begin{vmatrix}
    \;(\qquad e_1^\transp \qquad)\;\\[0.2em]
    \cdots\\[0.2em]
    \;(\qquad e_n^\transp \qquad)\;\\[0.6em]
    \;(\qquad n^\transp \qquad)\;
    \end{vmatrix}
$$

Строки матрицы -- компоненты ОНБ $\Rightarrow J(x_0)\neq 0$.\\
Из соображений непрерывности $J \neq 0$ также в некоторой окрестности точки $x_0$.\\
Значит, это диффеоморфизм класса $C^2$.

В новых переменных: $\displaystyle\sum_{k,l=1}^n
\hat{a}_{kl}(y) \pd{\hat{u}}{y_k}{y_l}
+ \hat{F} (y, \hat{u}, \nabla_y \hat{u}) = 0$

Условие характеристичности: \ $\displaystyle 0 = \hat{a}_{nn} = \sum\limits_{i,j=1}^{n} a_{ij}[x(y)]\pd{y_n}{x_i}\pd{y_n}{x_j}, \quad \text{ где } y_n(x) \equiv \omega(x)$

\imaginarySubsection{Характеристическая поверхность}
\begin{definition}[Характеристическая точка]
    Пусть дважды гладкая поверхность $S$ задана уравнением $\omega(x) = 0. \quad (\omega \in C^2;\ \nabla\omega \neq 0)$\\
    Тогда точка $x_0 \in S$ называется характеристической, если в этой точке $\ \displaystyle\sum\limits_{i,j=1}^{n} a_{ij}\, \omega_{x_i}\, \omega_{x_j} = 0$
\end{definition}

\begin{definition}[Характеристическая поверхность]
    Дважды гладкая поверхность называется {\bf характеристикой}, если все её точки характеристические.
\end{definition}
Заметим, что если поверхность $\omega(x) = 0$ -- характеристическая, то все поверхности $\omega(x) = \text{const}$ тоже характеристические. Поэтому характеристики образуют семейства.
\begin{center}
    \includegraphics[height=0.22\textwidth]{./pic 3_2.pdf}
\end{center}
\begin{example}
$\pd[2]{u}{x} - \pd[2]{u}{y} = 0.\quad S$ -- прямая $x=y, \quad \vec{n} = \left(-\dfrac1{\sqrt{2}}, \dfrac1{\sqrt{2}}\right);\quad \pd{u}{\vec{n}} = -\dfrac{u_x}{\sqrt2} + \dfrac{u_y}{\sqrt2} = u_1$

Производная $\dfrac{du_1}{d\vec l} = \left(\vec{l},\; \nabla\,\pd{u}{\vec{n}}\right) = \dfrac1{\sqrt{2}}\pd{}{x}\left[-\dfrac1{\sqrt2}u_x + \dfrac1{\sqrt{2}}u_y\right] + \dfrac1{\sqrt{2}}\pd{}{y}\left[-\dfrac1{\sqrt2}u_x + \dfrac1{\sqrt{2}}u_y\right] = 0.$

Значит, любую функцию на $S$ задать нельзя; прямая $x=y$ -- характеристика.
\end{example}
\vspace{5pt}


\begin{example}
$u_{tt}-a^2\Delta_xu = f(t,x),\ x\in\R^3\ $(волновое уравнение).

Характеристическое уравнение: $\brk*{\pd{\omega}{t}}^2 - a^2(\grad \omega)^2 = 0$

Этому уравнению удовлетворяет $\omega(t,x) = \underbrace{a^2t^2 - \vec{x}^2 = 0}_{\text{конус}}$.
\end{example}



\begin{example}
$u_{t}-a^2\Delta_xu = f(t,x)\ $(уравнение теплопроводности).

Характеристическое уравнение: $-a^2[\omega_{x_1}^2 + \dots + \omega_{x_n}^2] = 0\ \Rightarrow \ \omega_{x_i} = 0 \Forall i=\overline{1,n}$

Так как $\nabla \omega \not= 0 $, мы требуем $\omega_t \not=0$. 

Подходит $\omega(t,x) = t-C = 0 \qquad   \Rightarrow \qquad $характеристики --- это гиперплоскости $t=C$
\end{example}



\begin{example}
$\Delta u(x) = f(x)\ \text{(ур-е Пуассона)},\ x\in\R^n;$

Характеристическое уравнение: $\; a^2[\omega_{x_1}^2+\dots+\omega_{x_n}^2] = 0 \ \Rightarrow\ \omega_{x_i} = 0 \Forall i=\overline{1,n}.$

А мы требовали $\nabla \omega \neq 0\ \Rightarrow\ $ у уравнения эллиптического типа нет характеристик.
\end{example}

\Subsection{Теорема Ковалевской}

Функция $u(\vec{x}) = u(x_1,\dots,x_n)$ называется {\bf вещественно-аналитической} в $\vec{x}_0$, если в некоторой $U_{\varepsilon}(\vec{x}_0)$ она представима в виде 
$$u(\vec{x}) = \displaystyle\sum\limits_{|\alpha|>0}u_{\alpha}(\vec{x}-\vec{x}_0)^{\alpha},$$
где $\alpha$ -- мультииндекс, \quad $u_{\alpha} \in \R, \qquad (\vec{x}-\vec{x}_0)^{\alpha} = (x_1-x_1^0)^{\alpha_1}\cdot\dots\cdot(x-x_n^0)^{\alpha_n}$

\begin{theorem}[Ковалевской]
Пусть в уравнении $\displaystyle\sum\limits_{i,j=1}^{n} a_{ij}(\vec{x})\frac{\partial^2u}{\partial x_i\partial x_j} + F(\vec{x}, u, \nabla u) = 0$:
\begin{itemize}[nolistsep, noitemsep]
\item все $a_{ij}(\vec{x})$ -- вещественно-аналитические в $\vec{x}_0$ 
\item $F(\vec{x}, u, \nabla u)$ -- вещественно-аналитическая в $(\vec{x}_0, u_0(\vec{x}_0), \nabla u(\vec{x}_0))$ соответственно 
\item $\omega(\vec{x})$  -- вещественно-аналитическая в $\vec{x}_0$
\item $\vec{x}_0$ -- не характеристическая точка поверхности
\item $u_0,\ u_1$ -- вещественно-аналитические в $\vec{x}_0$.
\end{itemize}
Тогда: 
\begin{itemize}[nolistsep, noitemsep]
\item $\exists\: U_{\varepsilon}(\vec{x}_0)\colon$ в ней $\exists$ вещественно-аналитическое решение задачи Коши
\item оно единственно в классе вещественно-аналитических функций.
\end{itemize}
\end{theorem}

\end{document}

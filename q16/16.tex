% Затехал Хасянов Расул
\section{Билет 16. Решение методом Фурье задачи Дирихле для уравнения Лапласа в круге. Представление решения в виде ряда по однородным гармоническим многочленам и в виде интеграла Пуассона. Существование классического решения при непрерывной граничной функции.} 
\subsection{Решение методом Фурье задачи Дирихле для уравнения Лапласа в круге.}
Задача: в круге $D = \curlyBr{x| \abs{x} < R}$ и на границе $\Gamma =\partial D$ рассматриваем задачу:
\begin{equation} \label{16_1}
\begin{cases}
&\Delta u(x) = f(x), x \in D \longleftarrow \text{уравнение Пуассона}\\
&u|_\Gamma = u_0(x), x \in \partial D
\end{cases}
\end{equation}
Сделаем замену: 
$x_1 = \rho \cos \varphi, x_2 = \rho \sin \varphi$.\\ Функция $\hat u(\rho,\varphi) = u(\rho \cos \varphi, \rho \sin \varphi)$. \\Аналогично $\hat u_0(\rho, \varphi) = u_0(\rho \cos \varphi, \rho \sin \varphi), \hat f(\rho, \varphi) = f(\rho \cos \varphi, \rho \sin \varphi).$\\
Задача перепишется в виде: 
\[
\begin{cases}
&\hat u_{\rho \rho} + \frac{1}{\rho} \hat u_\rho + \frac{1}{\rho^2} \hat u_{\varphi \varphi} = \hat f(\rho, \varphi), 0 < \rho < R, 0 \leq \varphi \leq 2\pi\\
&\hat u (R,\varphi) = \hat u_0(R, \varphi)\\
&\hat u(\rho,\varphi) = \hat u(\rho, \varphi + 2\pi)
\end{cases}
\]
Далее считаем $\hat f = 0$, то есть решаем уравнение Лапласа.\\
Предположение: $u_0(x) \in C^1(\Gamma)$. Предполагаем, что решение принадлежит классу $C^2(D) \cap C^1(\bar D)$. При этом $\hat u$ -- $2\pi$-периодическая по $\varphi \Rightarrow$ можно разложить $\hat u_0, \hat u$ в ряды Фурье. [Если был было уравнение Пуассона -- требовали бы  $f \in C^1(\bar D)$.]
\begin{align*}
\curlyBr{\begin{matrix} \hat u(\rho,\varphi) \\ \hat u_0(R,\varphi) \end{matrix}} &= \frac{1}{2} \curlyBr{\begin{matrix} a_0(\rho) \\ A_0 \end{matrix}} + \sum_{k=1}^{\infty} \squareBr{\curlyBr{\begin{matrix} a_k(\rho) \\ A_k \end{matrix}} \cos k \varphi + \curlyBr{\begin{matrix} b_k(\rho) \\ B_k \end{matrix}} \sin k \varphi}, \text{где}\\
\curlyBr{\begin{matrix} a_k(\rho)\\ A_k \end{matrix}} &= \frac{1}{\pi} \int_0^{2\pi} \curlyBr{\begin{matrix} \hat u(\rho,\psi) \\ \hat u_0(R,\psi) \end{matrix}} \cos k \psi d \psi, k \in \N_0,\\ 
\curlyBr{\begin{matrix} b_k(\rho)\\ B_k \end{matrix}} &= \frac{1}{\pi} \int_0^{2\pi} \curlyBr{\begin{matrix} \hat u(\rho,\psi) \\ \hat u_0(R,\psi) \end{matrix}} \sin k \psi d \psi, k \in \N.
\end{align*}
Формально подставляем в уравнение(аргументы функций опускаем -- они все уже определены).
\[
\frac{1}{2}\roundBr{a''+\frac{1}{\rho}a_0'} + \sum_{k=1}^\infty \curlyBr{\squareBr{a_k''+\frac{1}{\rho}a_k'-\frac{k^2}{\rho^2}a_k} \cos k \varphi + \squareBr{b_k''+\frac{1}{\rho}b_k'-\frac{k^2}{\rho^2}b_k} \sin k \varphi} = 0
\]
Из граничного условия: 
\[
\frac{1}{2}a_0(R) + \sum_k \squareBr{a_k(R) \cos k \varphi + b_k(R) \sin k \varphi} = \frac{1}{2} A_0 + \sum_k \squareBr{A_k \cos k \varphi + B_k \sin k \varphi}
\]
В силу ортогональности тригонометрической системы в $L_2[0,2\pi]$ имеем на $a_k$ и $b_k$ следующие задачи:
\[
\begin{cases}
&a_k''(\rho)+ \frac{1}{\rho} a_k'(\rho) - \frac{k^2}{\rho^2} a_k(\rho) = 0, 0 \leq \rho \leq R\\
&a_k(R) = A_k, \ k \in \N_0\\
\end{cases} \qquad \begin{cases}
&b_k''(\rho)+ \frac{1}{\rho} b_k'(\rho) - \frac{k^2}{\rho^2} b_k(\rho) = 0, 0 \leq \rho \leq R\\
&b_k(R) = B_k, \ k \in \N\\
\end{cases}
\]
Будем искать только ограниченные решения -- для этого одного граничного условия окажется достаточно.\\
Решения данных уравнений Эйлера ищем в виде $\alpha \rho^\mu$:\\
Для первой серии задачи: $\alpha[\mu(\mu-1)+\mu - k^2] \rho^{\mu-2} = 0 \Rightarrow \mu = \pm k$.\\
Общее решение:
$$ a_k(\rho) = C_{1k}\rho^k+C_{2k}\rho^{-k} $$
$$a_0(\rho) = C_{10} \cdot 1 + C_{20} \cdot \ln \rho$$
Для ограниченности в круге берем $C_{2k} = C_{20} = 0 \Rightarrow a_k = C_{1k} \rho^k, C_{1k} = \frac{A_k}{R^k}, k \in \N_0$.\\
Итак, 
\begin{equation}\label{16_2}
\curlyBr{\begin{matrix} a_k(\rho)\\ b_k(\rho) \end{matrix}} = \curlyBr{\begin{matrix} A_k\\ B_k \end{matrix}}\roundBr{\frac{\rho}{R}}^k \Rightarrow \hat u(\rho, \varphi) = \frac{A_0}{2} + \sum_{k=1}^{\infty}\roundBr{A_k \cos k \varphi + B_k \sin k \varphi} \roundBr{\frac{\rho}{R}}^k  
\end{equation}
Вернемся в исходные переменные. Заметим, что если ввести $z = x_1+ix_2 = \rho e^{i \varphi}$, то $\rho^k \cos k \varphi = \mathrm{Re} \ z^k,\\ \rho^k \sin k \varphi = \mathrm{Im} \ z^k$.\\ Обозначим $\mathrm{Re} \ z^k = p_k(x_1,x_2), \mathrm{Im} \ z^k = q_k(x_1,x_2)$.
\begin{equation} \label{16_3}
\Rightarrow u(x_1,x_2) = \frac{A_0}{2}+ \sum_{k=1}^{\infty}\squareBr{\frac{A_k}{R^k} p_k(x_1,x_2) + \frac{B_k}{R^k} q_k(x_1,x_2)} 
\end{equation}
\subsection{Представление решения в виде ряда по однородным гармоническим многочленам и в виде интеграла Пуассона. Существование классического решения при непрерывной граничной функции.}
\begin{theorem}
Пусть $u_0 \in C(\Gamma)$. Тогда:
\begin{enumerate}
\item Существует и единственно классическое решение $u(x) \in C^\infty(D) \cap C(\bar D)$ задачи \ref{16_1} с $f \equiv 0$.
\item В $D$ это решение представимо рядами \ref{16_2} и \ref{16_3}, сходящимися в $|x| \leq R_1 < R$ равномерно.
\item Классическое решение представимо формулой Пуассона: \[u(x) = \frac{1}{2\pi R} \oint_{\partial D} \frac{R^2 - |x|^2}{|x- \xi|^2}u_0(\xi) dS_\xi\]
\item Любые частные производные по $x_1$ и $x_2$ вычисляются почленным дифференцированием ряда.
\end{enumerate}
\end{theorem}
\begin{proof}
\begin{enumerate}
\item Единственность докажем позднее (в этом билете ее нет).
\item Пусть $|u_0| < M$ на $\Gamma$. Тогда: 
\[
\abs{A_k} \leq \frac{1}{\pi}\int_0^{2\pi} \abs{u_0} \abs{\cos k \psi} d \psi \leq 2M, \; \abs{B_k} \leq 2M
\]
Запишем 
\[u(x_1,x_2) = \frac{A_0}{2}+ \sum_{k=1}^{\infty}\squareBr{\frac{A_k}{R^k} p_k(x_1,x_2) + \frac{B_k}{R^k} q_k(x_1,x_2)}  = \mathrm{Re}\, w_1 + \mathrm{Im}\, w_2, \text{где}\]
\[
w_1 = \frac{A_0}{2}+ \sum_{k=1}^{\infty} \frac{A_k}{R_k}z^k,\; w_2 = \sum_{k=1}^{\infty} \frac{B_k}{R_k}z^k
\]
Оба ряда сходятся абсолютно и равномерно в круге $|z| \leq R_1 < R \Rightarrow$ порождают в круге радиуса $R_1$ регулярные функции, что и требовалось.
\item Докажем формулу Пуассона: 
\begin{align*}
\hat u(\rho, \varphi) &= \frac{1}{2}\cdot \frac{1}{\pi} \int_0^{2\pi} \hat u_0(R,\psi) d \psi + \sum_{k=1}^{\infty} \frac{1}{\pi} \int_0^{2\pi} \hat u_0(\psi)\roundBr{\cos k \psi \cdot \cos k \varphi + \sin k \psi \sin k \varphi}d\psi \roundBr{\frac{\rho}{R}}^k =  \\
&= \roundBr{\text{сходимость равномерная}} = \frac{1}{2\pi}\int_0^{2\pi} \squareBr{1+ \sum_{k=1}^\infty 2\cos k(\psi - \varphi) \roundBr{\frac{\rho}{R}}^k}\hat u_0(\psi) d \psi = \\
& = \frac{1}{2\pi}\int_0^{2\pi} \squareBr{\sum_{k=0}^{\infty} e^{-ik(\psi-\varphi)}\roundBr{\frac{\rho}{R}}^k + \sum_{k=1}^{\infty} e^{ik(\psi-\varphi)}\roundBr{\frac{\rho}{R}}^k} \hat u_0(\psi) d \psi = \frac{1}{2\pi}\int_0^{2\pi} \squareBr{\sum_{k=0}^{\infty} p^k + \sum_{k=1}^{\infty} \bar p^k} \hat u_0(\psi) d \psi 
\end{align*}
Под интегралом ($|p| = |\bar p| = \frac{\rho}{R} < 1$): 
\begin{align*}
\sum_{k=1}^{\infty} p^k + \sum_{k=1}^{\infty} \bar p^k &= \frac{1}{1-p} + \frac{1}{1-\bar p} - 1 = \frac{1 - p + 1 - \bar p -1 +p + \bar p - p \bar p}{(1-p)(1-\bar p)} = \frac{1- \abs{p}^2}{1-(p+\bar p) + p\bar p} = \\
&=\frac{1- \abs{p}^2}{1-2 \mathrm{Re}\, p + \abs{p}^2} = \frac{1- \roundBr{\frac{\rho}{R}}^2}{1-2\frac{\rho}{R} \cos(\psi - \varphi) + \frac{\rho^2}{R^2}} = \frac{R^2 -\rho^2}{R^2 + \rho^2 -2R\rho\cos(\varphi - \psi)} = \\
&=\frac{R^2 - \abs{x}^2}{\abs{x-\xi}^2},\; \begin{matrix} x = (\rho, \varphi)\\ \xi = (R, \psi) \end{matrix}
\end{align*}
Итак, 
\begin{align*}
\hat{u}(\rho, \varphi) &= \frac{1}{2\pi R} \int_0^{2\pi} \frac{R^2 -\rho^2}{R^2 + \rho^2 -2R\rho\cos(\varphi - \psi)} \hat u_0(\psi) d \psi\\
\Longleftrightarrow u(x) &= \frac{1}{2\pi R}\oint_{\Gamma} \frac{R^2 - |x|^2}{|x - \xi|^2}u_0(\xi) dS_\xi
\end{align*}
Заметим, что при $u_0 \equiv 1$ мы получим ядро Пуассона:
\[
1 \equiv \frac{1}{2\pi R}\oint_{\Gamma} \frac{R^2 - |x|^2}{|x - \xi|^2} dS_\xi
\]
Покажем, что $u(x) \in C (\bar D)$.\\
Пусть $x \in D \cap \curlyBr{x : \abs{x-x^0} < \delta_{n_0}}$, где $x_0 \in \Gamma$, а $\delta_{n_0}$ выбрано так, чтобы $\abs{u_0(x) - u_0(x^0)} \leq \varepsilon$.
\begin{align*}
u(x) - u(x^0) &= \frac{1}{2\pi R}\oint_{\Gamma} \frac{R^2 - |x|^2}{|x - \xi|^2}\roundBr{u_0(\xi)-u_0(x^0)} dS_\xi = \\
&= \frac{1}{2\pi R} \roundBr{\int_{\curlyBr{\xi \in \Gamma: \abs{\xi - x^0} < \delta_{n_0}}} +  \int_{\curlyBr{\xi \in \Gamma: \abs{\xi - x^0} \geq \delta_{n_0}}}} \frac{R^2 - |x|^2}{|x - \xi|^2}\roundBr{u_0(\xi)-u_0(x^0)} dS_\xi = I_< + I_\geq
\end{align*}
\begin{align*}
\abs{I_<} \leq & \frac{1}{2 \pi R} \int_{(<)}  \frac{\abs{R^2 - |x|^2}}{|x - \xi|^2}\abs{u_0(\xi)-u_0(x^0)} dS_\xi \leq \varepsilon \cdot \curlyBr{\text{ядро Пуассона}} = \varepsilon\\
\abs{I_\geq} \leq & \frac{1}{2\pi R}\frac{(R - \abs{x})(R+\abs{x})}{\min \abs{\xi - x}^2} \cdot 2M \cdot \int_{(\geq)} dS_\xi \leq 4MR \frac{R - \abs(x) }{\min \abs{\xi - x}^2}, \text{где}\; R+|x| \leq 2R \text{ и} \int_{(\geq)} dS_\xi \leq 2 \pi R\\
& \abs{\xi - x} = \abs{\xi -x^0 + x^0 -x} \geq \abs{\xi - x^0} - \abs{x^0 -x} > \delta_{n_0} - \frac{\delta_{n_0}}{2} = \frac{\delta}{2} \Rightarrow \abs{I_\geq} \leq 16MR\frac{R - |x|}{\delta^2_{n_0}}\\
& \text{Возьмем } \delta_n = \min \curlyBr{\frac{\delta_{n_0}}{2}, \varepsilon \delta^2_{n_0}}. \text{ Тогда при } \abs{x-x^0} < \delta_n \text{ будем иметь } R- |x| \leq \abs{x^0 - x} < \delta_n  \Rightarrow \abs{I_\geq} \leq 16MR\varepsilon\;\;\;\;\;
\end{align*}
Итак, при $\abs{x^0 - x} < \delta_n: \abs{u(x) - u(x^0)} \leq \abs{I_<} + \abs{I_\geq} \leq \varepsilon + 16MR \varepsilon$. \\
Непрерывность доказана.
\item Дифференцируемость:\\ 
$u = \mathrm{Re}\, w_1 + \mathrm{Im}\, w_2$.\\
Вспомогательный факт: $\tilde{w}(z) = \tilde{u}(z) + i \tilde{v}(z) \Rightarrow \frac{d \tilde{w}}{d z} = \tilde{u}_{x_1} + i \tilde{v}_{x_1} \Rightarrow  \tilde{u}_{x_1} = (\mathrm{Re}\, \tilde{w})_{x_1} = \mathrm{Re}\, (\tilde{w}_z)$\\
В нашем случае: 
\[
\pd{}{x_1} \mathrm{Re}\, w_1(z) = \mathrm{Re}\, \frac{dw}{dz} = \mathrm{Re}\, \sum_{k=0}^{\infty} \frac{A_k}{R^k}\roundBr{z^k}'_z = \sum_{k=0}^{\infty} \frac{A_k}{R^k}\mathrm{Re}\, \frac{dz^k}{dz} = \sum_{k=0}^{\infty} \frac{A_k}{R^k}\pd{}{x_1}\mathrm{Re}\,z^k = \sum_{k=0}^{\infty} \frac{A_k}{R^k}\roundBr{p_k}'_{x_1}(x_1,x_2)
\]
Аналогично для $B_k$ и для производного любого порядка.
\end{enumerate}
\end{proof}
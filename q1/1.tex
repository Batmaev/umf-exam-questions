\documentclass[../main.tex]{subfiles}
\begin{document}

\section[Приведение уравнений второго порядка к каноническому виду]{
  Приведение к каноническому виду дифференциальных уравнений с частными производными второго порядка с линейной старшей частью в точке $x \in \R^n$. Классификация уравнений.}

% Затехал: К.А.А., 372
\imaginarySubsection{Приведение к каноническому виду в точке из \texorpdfstring{$\R^n$}{R\textasciicircum n}}

Пусть $\Omega \subset \R^{n}.$
ДУЧП 2 порядка с линейной старшей частью:
$$
\sum_{i,j=1}^n a_{ij}(x) \pd{u}{x_i}{x_j}
+ F(x,u,\nabla u) = 0    ; \qquad
u(x)      \in C^2(\Omega); \quad
a_{ij}(x) \in  C(\Omega)
$$
Считаем $a_{ij}(x) = a_{ji}(x)$, что не сужает класса, 
т.к. $u_{{x_i}{x_j}} = u_{{x_j}{x_i}}$.
Хотим сделать замену так, чтобы все смешанные частные производные обратились в 0. 
В точке это сделать можно.

Возьмём преобразование 
\[y = y(x) = 
\begin{cases} 
  y_1 = y_1(x_1, \dots, x_n) \\ 
  \dots                      \\ 
  y_n = y_n(x_1, \dots, x_n)
\end{cases} 
\in C^2(U(x^0)), \quad 
y^0 = y(x^0);    \quad 
U(x^0) \rightarrow V(y^0) 
\] 
(диффеоморфизм класса $C^2$ окрестности $U(x^0)$ на $V(y^0)$)

Будем предполагать $\exists$ обратного: $x = x(y)$.
% У диффеоморфизма вроде обратное преобразование $exists$ по определению...
Наша функция: $u = u(x_1, \dots, x_n)$. 

Введём $\hat{u}(y) \coloneqq u[x(y)] \in C^2(V(y^0))$.
Производные: 
$$
\pd{u}{x_i} 
= \sum_{k=1}^n       \pd{\hat{u}}{y_k}        \pd{y_k}{x_i};
\qquad
\pd{u}{x_i}{x_j} 
= \sum_{k,l=1}^n     \pd{\hat{u}}{y_k}{y_l}  \pd{y_k}{x_i}      \pd{y_l}{x_j}
+ \underbrace{
  \sum_{k=1}^n       \pd{\hat{u}}{y_k}        \pd{y_k}{x_i}{x_j}
  }
  _{  \text{уйдёт в }  \hat{F}(y,\hat{u},\nabla_y\hat{u})  }
$$
Подставляем: 
$$\sum_{i,j=1}^n     a_{ij}(x(y))  
  \sum_{k,l=1}^n     \pd{\hat{u}}{y_k}{y_l}  \pd{y_k}{x_i}      \pd{y_l}{x_j} 
+ \hat{F}(y,\hat{u},\nabla_y \hat{u}) 
= 0 $$
$$\sum_{k,l=1}^n  \left[
      \sum_{i,j=1}^n a_{ij}(x(y))              \pd{y_k}{x_i}       \pd{y_l}{x_j}
  \right] 
  \pd{\hat{u}}{y_k}{y_l}  +  \hat{F}(y,\hat{u},\nabla_y\hat{u})
= 0,$$
$$\hat{a}_{kl}(y) \coloneqq 
  \sum_{i,j=1}^n     a_{ij}(x(y))              \pd{y_k}{x_i}        \pd{y_l}{x_j}
$$

Введём матрицы: 
$A(x^0)      = \norm!{ a_{ij}(x^0) }_{i,j=1}^n;   \qquad 
\hat{A}(y^0) = \norm!{ \hat{a}_{ij}(y^0) }_{i,j=1}^n.$

$J(x^0)      = \norm*{ \pd{y_i}{x_j}(x^0) }_{i,j=1}^n$
 -- в малой $U(x^0)$ задаёт преобразование $\hat{A}(y^0) = J(x^0) A(x^0) J(x^0)^\Transp$ 

$ A = A^\Transp 
\Rightarrow 
\hat{A} = \hat{A}^\Transp$. 
Вопрос в выборе $J$ такого, что $\hat{A}$ диагональна.

Пусть в $\R^n$ заданы элемент $h$ и квадратичная форма $\Phi(h).$

Введём 2 базиса: $
\begin{array}{c} 
  (e_1  \dots  e_n) \\
  (e'_1 \dots e'_n)
\end{array}$
%
В них $h \sim 
\begin{array}{c} 
  \xi  = (\xi_1,  \dots, \xi_n )^\Transp  \\
  \eta = (\eta_1, \dots, \eta_n)^\Transp
\end{array}; \quad
% 
\Phi \sim 
\begin{array}{c} 
  \norm{c_{ij}}    \\
  \norm{\hat{c}_{ij}}
\end{array}; \quad
% 
\Phi(h) = 
\begin{array}{c}
  \xi ^\Transp    C    \xi   \\
  \eta^\transp \hat{C} \eta
\end{array}$

Пусть $\xi = S\eta$. 
Тогда $\Phi(h) = \eta^\transp S^\Transp C S \eta 
               = \eta^\transp    \hat{C}    \eta 
  \ \rightarrow\ \hat{C} = S^\Transp C S.$

Существует такой базис, что

$\hat{C} = \operatorname{diag}(
  \underbrace{+1, +1, \dots, +1}_{p\text{ штук}}, 
  \underbrace{-1, -1, \dots, -1}_{q\text{ штук}}, 
  0, 0, \dots, 0)$

$\Phi(h) 
= \eta^2_1     + \dots + \eta^2_p 
- \eta^2_{p+1} - \dots - \eta^2_{p+q}$

В равенстве $\hat{A}(y^0) = J(x^0) A(x^0) J(x^0)^\Transp$ 
нужно взять $J(x^0) = S^\Transp$

Такие преобразования существуют, их много. 
Например, $y = y^0 + J(x^0)(x-x^0)$

В этих переменных уравнение принимает вид:
$$
  \pd[2]{\hat{u}}{y_1}     + \dots + \pd[2]{\hat{u}}{y_p}
- \pd[2]{\hat{u}}{y_{p+1}} - \dots - \pd[2]{\hat{u}}{y_{p+q}} 
+ \hat{F}(y,\hat{u},\nabla_y\hat{u}) 
= 0.$$

\Subsection{Классификация уравнений}
\begin{enumerate}
  \item \textit{Эллиптический тип}: $p = n$ или $q = n$
  \item \textit{Ультрагиперболический тип}: $p+q = n$
  \item \textit{Гиперболический тип}: $p = 1,\ q = n-1$
  \item \textit{Ультрапараболический тип}: $p + q < n$
  \item \textit{Параболический тип}: $q = 0,\ p = n-1$
\end{enumerate}

\begin{remark}
$\hat{A}(y^0) = J(x^0)A(x^0)J(x^0)^\Transp 
\ \Rightarrow\ 
\sign\det\brk!{ \hat{A}(y^0) } = \sign\det\brk!{ A(x^0) }$ 
\end{remark}

В случае $n = 2$ тип уравнения в точке определяется по знаку определителя:
\begin{enumerate}

  \item \textit{Эллиптический: } $\hat{A}(y^0) = \begin{pmatrix}
    \pm1 & 0 \\ 0 & \pm1
    \end{pmatrix} \rightarrow \abs!{\hat{A}(y^0)} = 1$

  \item \textit{Гиперболический:} $\hat{A}(y^0) = \begin{pmatrix}
    1 & 0 \\ 0 & -1
    \end{pmatrix} \rightarrow \abs!{\hat{A}(y^0)} = -1$

  \item \textit{Параболический:}\; $\abs!{\hat{A}(y^0)} = 0$.
  
\end{enumerate}

\end{document}

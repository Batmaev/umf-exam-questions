\section{Применение метода Дюамеля для решения неоднородного уравнения теплопроводности в $\R^n$. Существование классического решения.}
% Техал Витяй
Задача:
\begin{equation} \label{12_1}
\begin{cases}
	u_t(t,x) - a^2\Delta_xu(t,x) = 0, \quad & t > 0, \; x \in \R^n, \\
    \eval{u}_{t=0} = u_0(x) = \varphi_1(x_1) \ldots \varphi_n(x_n), \quad & \varphi_k(x_k) \in C(\R^1), \quad \abs*{\varphi(x_k)} \leqslant M, \quad k = \overline{1,n}.
\end{cases}
\end{equation}
Напишем серию задач Коши:
\def \uk {\stackrel{\text{\tiny{k}}}{u}}
\def \uj {\stackrel{\text{\tiny{j}}}{u}}
\begin{equation}
\begin{cases}
	\uk_t(t,x) - a^2\uk_{x_kx_k}(t,x) = 0, \\
    \eval{\uk}_{t=0} = \varphi_k(x_k).
\end{cases}
\end{equation}
Решение каждой даётся формулой Пуассона:
\begin{equation}
	\uk(t,x_k) = \dfrac{1}{\sqrt{4\pi a^2t}} \int\limits_{-\infty}^{+\infty} e^{-\frac{(x_k - y_k)^2}{4a^2t}}\varphi_k(y_k)dy_k
\end{equation}
Покажем, что решение всей задачи:
\begin{equation*}
	u = \prod_{k=1}^n \uk(t,x_k)
\end{equation*}
\begin{enumerate}
	\item Очевидно, $u(t,x) \in C(t \geqslant 0, x \in \R^n) \cap C^\infty(t > 0, x \in \R^n)$;
    \item $u(0,x) = \displaystyle\prod_{k=1}^n \varphi_k(x_k)$;
    \item $u_t(t,x) = \dfrac{\partial}{\partial t} \brk[s]*{\displaystyle\prod_{k=1}^n \uk(t,x_k)} = \displaystyle\sum_{j=1}^n \brk[s]*{\uj_t(t,x_j) \displaystyle\underset{k \not= j}{\prod_{k=1}^n} \uk(t,x_k)} = \displaystyle\sum_{j=1}^n \brk[s]*{a^2 \uj_{x_jx_j} \displaystyle\underset{k \not= j}{\prod_{k=1}^n} \uk(t,x_k)} = \\ = a^2 \displaystyle\sum_{j=1}^n \dfrac{\partial^2}{\partial x_j^2} \displaystyle\prod_{k=1}^n \uk(t,x_k) = a^2 \displaystyle \sum_{j=1}^n \dfrac{\partial^2}{\partial x_j^2} u(t,x) = a^2 \Delta_x u(t,x)$.
\end{enumerate}
Итак, в случае разделения переменных имеем:
\begin{equation}
u(t,x) = \brk*{\dfrac{1}{\sqrt{4\pi a^2t}}}^n \int\limits_\R\ldots\int\limits_\R e^{-\frac{\sum_{j=1}^n (x_j - y_j)^2}{4a^2t}} \prod_{k=1}^n \varphi_k(y_k) dy_1\ldots dy_n = \boxed{\brk*{\dfrac{1}{\sqrt{4\pi a^2 t}}}^n \int\limits_{\R^n} e^{-\frac{\abs*{x-y}^2}{4a^2t}} u_0(y)dy}
\end{equation}
Полученная формула называется {\bf формулой Пуассона в $\R^n$}.
\begin{theorem}
	Пусть $u_0 \in C(\R^n), \, \abs*{u_0} \leqslant M \Forall x \in \R^n$. Тогда 
    \begin{equation*}
    	u(t,x) = \brk*{\dfrac{1}{\sqrt{4\pi a^2 t}}}^n \int\limits_{\R^n} e^{-\frac{\abs*{x-y}^2}{4a^2t}} u_0(y)dy \text{ ---}
    \end{equation*}
    классическое решение задачи Коши (\ref{12_1}), лежащее в классе $C^\infty(t>0, x \in \R^n) \cap C(t \geqslant 0, x \in \R^n)$. Кроме того,~$\abs*{u(t,x)} \leqslant M \Forall t > 0 \Forall x \in \R^n$.
\end{theorem}
\begin{proof}
	Сохраняется из предыдущего билета с заменой $(x-y)^2 \to \abs*{x-y}^2$.
\end{proof}
\bigskip
%\newpage
Зная решение однородного уравнения, можно найти решение и для неоднородного:
\begin{equation}
	\begin{cases}
    	u_t - a^2\Delta_xu = f(t,x), \quad t > 0, \, x \in \R^n, \\
        \eval{u}_{t=0} = 0.
    \end{cases}
\end{equation}
Используем метод Дюамеля. Предположения относительно $f$:
\begin{enumerate}
	\item $D_x^\alpha f(t,x) \in C(t \geqslant 0, x \in \R^n) \Forall \alpha: \abs*{\alpha} \leqslant 2$;
    \item $\abs*{f(t,x)} \leqslant M_0 \Forall t \geqslant 0 \Forall x \in \R^n$;
    \item $\abs*{D_x^\alpha f(t,x)} \leqslant M_2 \Forall t \geqslant 0 \Forall x \in \R^n$.
\end{enumerate}
Сводим задачу к семейству однопараметрических задач:
\begin{equation}
	\begin{cases}
    	v_t(t,x,\tau) - a^2\Delta_xv(t,x,\tau) = 0, \quad & t > \tau, \, x \in \R^n, \\
        \eval{v}_{t = \tau} = f(\tau, x), \quad & x \in \R^n.
    \end{cases}
\end{equation}
Решение даётся формулой Пуассона в $\R^n$:
\begin{equation*}
	v(t,x,\tau) = \brk*{\dfrac{1}{\sqrt{4\pi a^2 (t - \tau)}}}^n \int\limits_{\R^n} e^{-\frac{\abs*{x-y}^2}{4a^2(t-\tau)}} f(\tau, y) dy
\end{equation*}
$v(t,x,\tau)$ непрерывно продолжима до $t \geqslant \tau$, ограничена: $\abs*{v(t,x,\tau)} \leqslant M_0$.

Покажем, что $u(t,x) = \displaystyle\int\limits_0^t v(t,x,\tau)d\tau$~--- решение задачи.

Исследуем вспомогательную функцию:
\begin{equation*}
	w(\tilde{t}, x, \tau) = \brk*{\dfrac{1}{\sqrt{4\pi a^2\tilde{t}}}}^n \int\limits_{\R^n} e^{-\frac{\abs*{x-y}^2}{4a^2\tilde{t}}} f(\tau,y) dy, \quad \tilde{t} > 0, \, \tau \geqslant 0, \, x, y \in \R^n
\end{equation*}
Введём $\eta = \dfrac{y-x}{2a\sqrt{\tilde{t}}} \Rightarrow y = x + 2a\sqrt{\tilde{t}} \eta \Rightarrow dy = dy_1\ldots dy_n = \brk*{2a\sqrt{\tilde{t}}}^n d\eta_1\ldots d\eta_n = \brk*{2a\sqrt{\tilde{t}}}^n d\eta$.

Тогда $w(\tilde{t}, x, \tau) = \dfrac{1}{\pi^{n/2}} \displaystyle\int\limits_{\R^n} e^{-\eta^2} f\brk*{\tau, x+2a\sqrt{\tilde{t}}\eta}d\eta, \quad \tilde{t} > 0$

$f\brk*{\tau, x+2a\sqrt{\tilde{t}}\eta} \in C(\tau \geqslant 0, \tilde{t} \geqslant 0, x \in \R^n, \eta \in \R^n)$

$\abs*{f\brk*{\tau, x+2a\sqrt{\tilde{t}}\eta}e^{-\eta^2}} \leqslant M_0 e^{-\eta^2}$ и $\displaystyle\int\limits_{\R^n} e^{-\eta^2}d\eta <+\infty \Rightarrow w(\tilde{t}, x, \tau)$ сходится равномерно.

Итак, $w \in C(\tilde{t} \geqslant 0, \tau \geqslant 0, x \in \R^n)$.

\begin{statement}
	$w$ можно дифференцировать и вносить производную под интеграл.
\end{statement}

\begin{proof}
	$w_{x_i}(\tilde{t}, x, \tau) \sim \dfrac{1}{\pi^{n/2}} \displaystyle\int\limits_{\R^n} e^{-\eta^2} f_{x_i}\brk*{\tau, x+2a\sqrt{\tilde{t}}\eta}d\eta, \quad \tilde{t} > 0$

	$f_{x_i}\brk*{\tau, x+2a\sqrt{\tilde{t}}\eta} \in C(\tau \geqslant 0, \tilde{t} \geqslant 0, x \in \R^n, \eta \in \R^n)$

	$\abs*{f_{x_i}\brk*{\tau, x+2a\sqrt{\tilde{t}}\eta}e^{-\eta^2}} \leqslant M_2 e^{-\eta^2} \Rightarrow w_{x_i}(\tilde{t}, x, \tau)$ сходится равномерно.

	Поэтому, вместо '$\sim$' можно поставить '$=$':

	$w_{x_i}(\tilde{t}, x, \tau) = \dfrac{1}{\pi^{n/2}} \displaystyle\int\limits_{\R^n} e^{-\eta^2} f_{x_i}\brk*{\tau, x+2a\sqrt{\tilde{t}}\eta}d\eta, \quad \tilde{t} > 0$

	Аналогично и для вторых производных по $x$:

	$w_{x_ix_j}(\tilde{t}, x, \tau) = \dfrac{1}{\pi^{n/2}} \displaystyle\int\limits_{\R^n} e^{-\eta^2} f_{x_ix_j}\brk*{\tau, x+2a\sqrt{\tilde{t}}\eta} d\eta, \quad \tilde{t} > 0$
\end{proof}
	
	Теперь производная по времени: в силу того, что уравнение
\begin{equation*}
	\begin{cases}
		w_{\tilde{t}} - a^2 \Delta_x w = 0, \\
		\eval{w}_{\tilde{t} = 0} = f(\tau, x);
	\end{cases}
\end{equation*}

	выполняется везде, включая границу, получаем, что $w_{\tilde{t}} \in C(\tilde{t} \geqslant 0, \tau \geqslant 0, x \in \R^n)$

	Мы исследовали $w$, а цель -- $v$. Связь этих функций: $v(t,x,\tau) = w(t-\tau,x,\tau)$. При условиях $\tau \geqslant 0, t \geqslant \tau, x \in \R^n$ имеем непрерывность следующих функций: $v, v_t, v_{x_i}, v_{x_ix_j}$. Тогда для функции $u(t,x) = \displaystyle\int\limits_0^t v(t,x,\tau)d\tau$ получаем непрерывность $u, u_t, u_{x_i}, u_{x_ix_j} \Rightarrow$ решение будет классическим.

	Осталось проверить уравнение: $u_t = v(t,x,t) + \displaystyle\int\limits_0^t v_t(t,x,\tau)d\tau = f(t,x) + \displaystyle\int\limits_0^t a^2 \Delta_x v d\tau = f(t,x) + a^2 \Delta_x u$

	\begin{definition}
		Пусть $Q$~--- область в $\R^{n+1}_{t,x_1,\ldots,x_n}$, а $\hat{Q} = Q \cup \{\text{некоторое подмножество $\partial Q$}\}$. Обозначим $C^{p,q}_{t,x}(\hat{Q})$ множество функций $u(t,x)$ таких, что $u, D_x^\alpha u, D_t^\beta u \in C(Q), \alpha\text{ -- мультииндекс, }\abs*{\alpha} \leqslant q, \, \beta \in \N \cup \{0\}, \, \beta \leqslant p$, и все эти функции допускают непрерывное продолжение на $\hat{Q}$.
	\end{definition}

	\begin{theorem}
		Пусть в задаче Коши
			\begin{equation*}
				\begin{cases}
					u_t - a^2 \Delta_x u = f(t,x), \quad & t > 0, \, x \in \R^n,\\
					\eval{u}_{t=0} = u_0(x), \quad & x \in \R^n;
				\end{cases}
			\end{equation*}
		\begin{itemize}
			\item[а)] $u_0 \in C(\R^n), \abs*{u_0(x)} \leqslant M_0 \Forall x \in \R^n$
			\item[б)] $f(t,x) \in C^{0,2}_{t,x}(t \geqslant 0, x \in \R^n)$ 
            \item[в)] $\abs*{f(t,x)}\leq M_1,\ \abs*{f_{x_i}(t,x)}\leq M_2,\ \abs*{f_{x_ix_j}(t,x)}\leq M_2\ \Forall t\geq 0 \Forall x\in\R^n$ 
		\end{itemize}
		Тогда функция $$u(t,x) = \brk*{\dfrac{1}{\sqrt{4\pi a^2t}}}^n \displaystyle\int\limits_{\R^n} e^{-\frac{\abs*{x-y}^2}{4a^2t}} u_0(y)dy + \displaystyle\int\limits_0^t \brk[s]*{\brk*{\dfrac{1}{\sqrt{4\pi a^2(t-\tau)}}}^n \displaystyle\int\limits_{\R^n} e^{-\frac{\abs*{x-y}^2}{4a^2(t-\tau)}} f(\tau,y) dy} d\tau$$
		является классическим решением задачи Коши, лежит в классе $C(t \geqslant 0, x \in \R^n) \cap C^{1,2}_{t,x}(t > 0, x \in \R^n)$. Кроме того, справедлива оценка $\abs*{u(t,x)} \leqslant M_0 + tM_1$.
	\end{theorem}
	\begin{proof}
		Последний факт: $\abs*{u} \leqslant M_0 + \displaystyle\int\limits_0^t \abs*{v_t} d\tau \leqslant M_0 + M_1t$.
	\end{proof}
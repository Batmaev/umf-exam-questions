% Шаблон Юрия Максимова
% Лекции Зубова, конспект лекций и билеты Павла Останина

\documentclass[12pt]{article}
\usepackage{graphicx}
%\pagestyle{empty}
\usepackage[T2A]{fontenc}
\usepackage[utf8]{inputenc}
\usepackage[english,russian]{babel}
%\usepackage{cmap}
\usepackage{amsthm}
\usepackage{amscd}
\usepackage{amsmath}
\usepackage{wasysym}
\usepackage{wrapfig}
\usepackage{datetime}
\usepackage{cancel}
\usepackage{mathtools}
\usepackage{units}
\usepackage{fancyhdr}
\usepackage{forloop}
\usepackage{amssymb}
\usepackage{url}
\usepackage{comment}
\usepackage[colorlinks = true, urlcolor = blue]{hyperref}
\usepackage{xcolor}
\usepackage[inline]{enumitem}
\usepackage{wrapfig}
\usepackage{cmll}
\usepackage{varwidth}
\usepackage{subfiles}
%s\pagenumbering{arabic}
% Pogodin theorem definitions
% \

% \newtheorem*{theorem}{Теорема}
% \newtheorem*{lemma}{Лемма}
% \newtheorem*{remark}{Замечание}
% %\newtheorem*{proof}{Доказательство} %уже есть
% \newtheorem*{statement}{Утверждение}
% \newtheorem*{definition}{Определение}
% \newtheorem*{example}{Пример}

% Ivanychev theorem definitions
\theoremstyle{plain}
\newtheorem{theorem}{Теорема}[section]
\newtheorem{lemma}[theorem]{Лемма}
\newtheorem{statement}[theorem]{Утверждение}
\newtheorem*{conseq}{Следствие}
\newtheorem{offtop}{Offtop}[section]

\theoremstyle{definition}
\newtheorem{definition}{Определение}[section]
\newtheorem{example}{Пример}[section]

\theoremstyle{remark}
\newtheorem*{remark}{Замечание}

\renewcommand{\thesection}{\arabic{section}}
%\renewcommand{\headrulewidth}{0.4pt}
%\renewcommand{\footrulewidth}{0.4pt}

%\pagestyle{empty}
%

\renewcommand{\baselinestretch}{1.0}
\renewcommand\normalsize{\sloppypar}

\setlength{\topmargin}{-0.5in}
\setlength{\textheight}{9.1in}
\setlength{\oddsidemargin}{-0.3in}
\setlength{\evensidemargin}{-0.3in}
\setlength{\textwidth}{7in}
\setlength{\parindent}{0ex}
\setlength{\parskip}{1ex}

\newcounter{problemset}
% \newcounter{example}
%\newcounter{totalpages}
%Here you should set the total number of pages
%\setcounter{totalpages}{1}

% Defines
\def \bR {\mathbb{R}} % Красивое R для действ. чисел
\def \subd {\partial }
\def \vec {\boldsymbol}

\newcommand{\while}[2]{\left. #1\right|_{#2}}
\newcommand{\brs}[1]{\left(#1\right)}
\newcommand{\sbrs}[1]{\left[#1\right]}
\newcommand{\fbrs}[1]{\left\{#1\right\}}
\newcommand{\rbrs}[1]{\left\langle #1 \right\rangle}

\newcommand{\brbr}[1]{\bigl(#1\bigr)}
\newcommand{\erbr}[1]{\biggl(#1\biggr)}
\newcommand{\bfbr}[1]{\bigl[#1\bigr]}
\newcommand{\efbr}[1]{\biggl[#1\biggr]}


\newcommand{\R}{\mathbb{R}}
\newcommand{\Q}{\mathbb{Q}}
\renewcommand{\C}{\mathbb{C}}
\newcommand{\N}{\mathbb{N}}
\newcommand{\Z}{\mathbb{Z}}
\renewcommand{\phi}{\varphi}
\renewcommand{\epsilon}{\varepsilon}
\newcommand{\sign}{\mathrm{sign}\;}

\newcommand{\Equiv}{\Leftrightarrow}
\newcommand{\To}{\Rightarrow}

\newcommand{\eps}{\varepsilon}
\let \oldcmdforall \forall
\renewcommand{\forall}{\;\oldcmdforall\:}
\let \oldcmdexists \exists
\renewcommand{\exists}{\;\oldcmdexists\:}
\renewcommand{\leq}{\leqslant}
\renewcommand{\geq}{\geqslant}

% \abs and \norm
\DeclarePairedDelimiter\abs{\lvert}{\rvert}%
\DeclarePairedDelimiter\norm{\lVert}{\rVert}%
\makeatletter
\let\oldabs\abs
\def\abs{\@ifstar{\oldabs}{\oldabs*}}
%
\let\oldnorm\norm
\def\norm{\@ifstar{\oldnorm}{\oldnorm*}}
\makeatother
% derivatives
\renewcommand{\d}[2]{\frac{d #1}{d #2}}
\newcommand{\nd}[3]{\frac{d^{#3} #1}{d #2^{#3}}}
\newcommand{\dd}[2]{\nd{#1}{#2}{2}}

\newcommand{\pd}[2]{\frac{\partial #1}{\partial #2}}
\newcommand{\spd}[3]{\frac{\subd^2 #1}{\subd #2\subd #3}}
\newcommand{\npd}[3]{\frac{\partial^{#3} #1}{\partial #2^{#3}}}
\newcommand{\pdd}[2]{\npd{#1}{#2}{2}}

% Позволяет писать \item'ы в строчку
% Пример:
% \begin{itemize}
% 	\item bla \inlineitem bla \inlineitem bla
%		== здесь перешли на новую строку ==
%	\item bla \inlineitem bla \inlineitem bla
% \end{itemize}
\makeatletter
\newcommand{\inlineitem}[1][]{%
\ifnum\enit@type=\tw@
    {\descriptionlabel{#1}}
  \hspace{\labelsep}%
\else
  \ifnum\enit@type=\z@
       \refstepcounter{\@listctr}\fi
    \quad\@itemlabel\hspace{\labelsep}%
\fi}

\makeatletter
\AddEnumerateCounter{\asbuk}{\russian@alph}{щ}
\makeatother

\usepackage{tikz}
\usetikzlibrary{patterns}

\title{Уравнения математической физики}
\date{\today \currenttime}
\author{Программа билетов к экзамену. Поток В.И. Зубова}

\usepackage{natbib}
\usepackage{graphicx}

\begin{document}


\maketitle

\emph{Конспект подготовлен на основе лекций В.И. Зубова и подготовленных билетов Павла Останина и Михаила Христиченко. Делали:} \\
\ \\
\ \\
\ \\
\ \\
\begin{minipage}{0.1\textwidth}
\end{minipage}
\begin{minipage}{1\textwidth}
\centering
    \begin{varwidth}{\textwidth}
      \begin{itemize}
          \item Иванычев Сергей, 376 группа
          \item Погодин Роман, 374 группа
          \item Нагайко Иван, 372 группа
          \item Рязанов Андрей, 374 группа
          \item Федоряка Дмитрий, 374 группа
          \item Багно Богдан, 376 группа
          \item Изутин Никита, 378 группа
          \item Ермолова Марина, 373 группа
          \item Хасянов Расул, 371 группа
          \item Михальченко Егор, 371 группа
          \item Шлёнский Владислав, 374 группа
          \item Цветкова Ольга, 374 группа
          \item Молибог Игорь, 374 группа
          \item Чигринский Виктор, 374 группа
          \item Леонтьев Семён, 377 группа
          \item Кильянов Александр, 372 группа \item Тернов Лёха, 228 группа
    \end{itemize}
  \end{varwidth}
\end{minipage}
\begin{minipage}{0.1\textwidth}
\end{minipage}
\newpage
\tableofcontents


%\subfile{biblio.tex}

\newpage
\subfile{q1/1.tex}
\subfile{q1/1_4.tex}
\newpage
\subfile{q2/2.tex}
\newpage
\subfile{q3/3.tex}
\newpage
% Сергей Иванычев 
\subfile{q4/4.tex}
\newpage
\subfile{q5/5.tex}
\newpage
% Сергей Иванычев 
\subfile{q6/6.tex}
\newpage
\subfile{q7/7.tex}
\newpage
\subfile{q8/8.tex}
\newpage
\subfile{q9/9.tex}
\newpage
\subfile{q10/10.tex}
\newpage
\subfile{q11/11.tex}
\newpage
\subfile{q12/12.tex}
\newpage
\subfile{q13/13.tex}
\newpage
\subfile{q14/14.tex}
\newpage
\subfile{q15/15.tex}
\newpage
\subfile{q16/16.tex}
\newpage
\subfile{q17/17.tex}
\newpage
\subfile{q18/18.tex}
\newpage
\subfile{q19/19.tex}
\newpage
\subfile{q20/20.tex}
\newpage
\subfile{q21/21.tex}
\newpage
\subfile{q22/22.tex}
\newpage
\subfile{q23/23_1.tex}
\subfile{q23/23_2.tex}
\subfile{q23/23_3.tex}
\newpage
\subfile{q24/24.tex}
\newpage
\subfile{q25/25.tex}
\newpage
\subfile{q26/26.tex}
\newpage
\subfile{q27/27.tex}
\newpage
\subfile{q28/28.tex}
\newpage
\subfile{q29/29.tex}
\newpage
\subfile{q30/30cut.tex}
\newpage
\subfile{q31/31.tex}
\newpage
\subfile{q32/32.tex}
\end{document}

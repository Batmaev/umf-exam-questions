\section{Интегральные уравнения Фредгольма второго рода с вырожденными ядрами. Сведение их к системе линейных алгебраических уравнений. Теоремы Фредгольма в этом случае.}
% Затехал: Багно Богдан
Рассмотрим уравнение
$$u(x) = \lambda \int_{G}\mathcal{K}(x,y)u(y)dy + f(x), \; x \in \overline{G} \eqno(1)$$
\begin{definition}
Интегральное уравнение вида
$$v(x) = \lambda \int_{G}\mathcal{K}'(x,y)v(y)dy + g(x), \; x \in \overline{G}, \; \mathcal{K}'(x,y) = \mathcal{K}(y,x) $$
называется союзным уравнению (1).
\end{definition}
\begin{definition}
Ядро $\mathcal{K}(x,y) \in C(\overline{G} \times \overline{G})$ называется вырожденным, если оно представимо в виде
$$\mathcal{K}(x,y) = \sum_{i = i}^{N}a_{i}(x)b_{i}(y), \; a_{i},b_{i} \in C(\overline{G})$$
\end{definition}

Будем считать что $\brk[c]*{a_{1},...,a_{n}}$ и $\brk[c]*{b_{1},...,b_{n}}$ -- линейно независимые наборы (если это не так, то уменьшим $N$). Будем теперь рассматривать уравнение
$$u(x) = \lambda \int_{G}\brk[s]*{\sum_{i = 1}^{N}a_{i}(x)b_{i}(y)}u(y)dy + f(x), \; x \in \overline{G} \eqno(2)$$
\begin{itemize}
  \item Введем в $C(\overline{G})$ билинейную форму $\brk[a]*{u;v} = \int_{G}u(x)v(x)dx \Forall u,v \in C(\overline{G})$
  \item Введем также следующие обозначения:
  \begin{itemize}
    \item $\mu_{ij} = \brk[a]*{b_{i}; a_{j}}; \; A = \norm*{\mu_{ij}}_{i,j}^{N}$
    \item $\varphi_{i} = \brk[a]*{b_{i}; f}; \; \vec{\varphi} = \norm*{\varphi_{1},...,\varphi_{N}}^{T}$
    \item $c_{i} = \brk[a]*{b_{i}; u}; \; \vec{c} = \norm*{c_{1},...,c_{N}}^{T} \; (3)$
  \end{itemize}
  \item
    \begin{lemma}[об эквивалентности]
      Пусть $u(x) \in C(\overline{G})$ -- решение уравнения (2). Тогда $u(x) = \lambda \sum_{i=1}^{N}c_{i}a_{i}(x) + f(x), \; x \in \overline{G}$, где $\vec{c}$ определяется (3) и удовлетворяет системе $(E - \lambda A)\vec{c} = \vec{\varphi}$. Обратно, если $\vec{c}$ -- некоторое решение системы $(E - \lambda A)\vec{c} = \vec{\varphi}$ то $u(x) = \lambda \sum_{i=1}^{N}c_{i}a_{i}(x) + f(x), \; x \in \overline{G}$ является решением интегрального уравнения.
    \end{lemma}
    \begin{proof}
      \begin{enumerate}
        \item Пусть $u(x)$ решение интегрального уравнения. Тогда
      $$u(x) = \lambda \int_{G}\brk[s]*{\sum_{j = 1}^{N}a_{j}(x)b_{j}(y)}u(y)dy + f(x) = \lambda \sum_{j = 1}^{N}a_{j}(x)c_{j} + f(x)$$
      Домножим на $b_{i}$ и проинтегрируем по $G$:
      $$c_{i} = \lambda \sum_{j = 1}^{N} \mu_{ij}c_{j} + \varphi_{i} \Longleftrightarrow \vec{c} = \lambda A \vec{c} + \vec{\varphi}$$
      \item Обратно, если $c_{i} = \lambda \sum_{j = 1}^{N} \mu_{ij}c_{j} + \varphi_{i}, \; i = \overline{1,N}$ то рассмотрим $u_{*}(x) = \lambda \sum_{j = 1}^{N}a_{j}(x)c_{j} + f(x)$. Подставим в уравнение:
      $$u_{*}(x) - \lambda \int_{G}\mathcal{K}(x,y)u_{*}(y)dy - f(x) =$$
      $$= u_{*}(x) - \lambda \int_{G}\brk[s]*{\sum_{j = 1}^{N}a_{j}(x)b_{j}(y)}u_{*}(y)dy - f(x) =$$
      $$ = \lambda \sum_{j = 1}^{N}a_{j}(x)\brk[s]*{c_{j} - \int_{G}b_{j}(y)u_{*}(y)dy} =$$
      $$= \lambda \sum_{j = 1}^{N}a_{j}(x)\brk[s]*{c_{j} - \lambda\sum_{i = 1}^{N}c_{i} \underbrace{\int_{G}b_{j}(y)a_{i}(y)dy}_{\mu_{ji}} - \underbrace{\int_{G}b_{j}(y)f(y)dy}_{\varphi_{i}}} =$$
      $$ = \lambda \sum_{j = 1}^{N}a_{j}(x)\underbrace{\brk[s]*{c_{j} - \lambda\sum_{i=1}^{N}\mu_{ji}c_{i} - \varphi_{j}}}_{0}$$
      \end{enumerate}
    \end{proof}
\end{itemize}

Таким образом, исследование интегральных уравнений с вырожденным ядром эквивалентно иссследованию системы $(E -\lambda A)\vec{c} = \vec{\varphi}$.

\begin{offtop}
Отметим что для союзного уравнения $v(x) = \lambda \sum_{j = 1}^{N}b_{j}(x)\underbrace{\int_{G}a_{j}(y)v(y)dy}_{d_{j}} + g(y)$ соответствующей системой является $(E - \lambda A^{T})\vec{d}=\vec{\phi}$, где $\vec{\phi}=\int_{G} \vec{a}(y)g(y)dy$.
\end{offtop}

\Subsection{Разрешимость интегрального уравнения с вырожденным ядром}
Пусть $D(\lambda) = \mathrm{det}\,(E - \lambda A) = \mathrm{det}\,(E - \lambda A^{T})$. Ясно что $D(\lambda) \not\equiv 0$ т.к. $D(0) = 1$. $D(\lambda)$ есть многочлен $P(\lambda), \; \mathrm{deg}\, P \leq N \longrightarrow$ он имеет $p$ действительных корней $\lambda_{1},...,\lambda_{p},\; 0 \leq p \leq N$.
\begin{itemize}
  \item Если $D(\lambda) \neq 0$ то $\forall k: \lambda_{k} \neq \lambda \longrightarrow$ у уравнения $(E - \lambda A)\vec{c} = \vec{\varphi}$ решение существует и оно единственно. Аналогичное утверждение верно и для союзного уравнения.
  \item Если $\exists k: \lambda_{k} = \lambda \rightarrow \mathrm{rk}\,(E - \lambda A) = \mathrm{rk}\,(E - \lambda A^{T}) = r < N$
\end{itemize}

Пусть $m = N - r > 0$. Тогда базис в пространстве решений $(E - \lambda A)\vec{c}$ обозначим как $\vec{c}_{1},...,\vec{c}_{N}$ а базис в пространстве решений $(E - \lambda A^{T})\vec{d}=0$ обозначим как $\vec{d}_{1},...,\vec{d}_{N}$. Соответствующие им решения обозначим $u_{1},...,u_{m}$ и $v_{1},...,v_{m}$ соответственно $(u_{k}(x) = \lambda \sum_{j = 1}^{N}a_{j}(x)c_{j,k}, \; v_{k}(x) = \lambda \sum_{j = 1}^{N}b_{j}(x)d_{j,k})$.

Покажем что $u_{1},...,u_{N}$ базис решения однородного уравнения. Пусть
$$\sum_{i=1}^{m}\alpha_{i}u_{i} = 0 \Longleftrightarrow \sum_{i=1}^{m}\alpha_{i}\brk[s]*{\lambda\sum_{j=1}^{N}a_{j}(x)c_{j,i}} = 0 \Longleftrightarrow \sum_{i=1}^{m}\alpha_{i}c_{i,j} = 0 \Longleftrightarrow \sum_{i=1}^{m}\alpha_{i}\vec{c}_{i} = 0$$
Значит система $u_{1},...,u_{m}$ -- линейно независима.

\begin{definition}[Собственные функции и собственные числа оператора $K$]
Функция $u(x) \in C(\overline{G}), \; u \not\equiv 0$ удовлетворяющая уравнению
$$u(x) = \lambda \int_{G}\mathcal{K}(x,y)u(y)dy, \; x \in C(\overline{G})$$
называется собственной функцией ядра $\mathcal{K}$ или собственной функцией оператора $K$. Соответствующие собственным функциям $\lambda$ называются характеристическими числами ядра/оператора $K$.
\end{definition}

Свойства характеристических чисел:
\begin{itemize}
  \item $\lambda \neq 0$ (иначе $u \equiv 0$).
  \item $\lambda$ -- не собственное значение оператора: $u = \lambda K u \Leftrightarrow Ku = \frac{1}{\lambda}u => \Rightarrow \mu = \frac{1}{\lambda}$ -- собственное значение.
  \item Собственные значения $K$ и $K'$ совпадают.
\end{itemize}

Рассмотрим систему $(E - \lambda A)\vec{c} = \vec{\varphi}$.
По теореме Фредгольма из линейной алгебры (2 сем) эта система совместна тогда и только тогда, когда каждое решение сопряженной однородной системы $(E - \lambda A^{T})\vec{d} = \vec{0}$ ортогонально $\vec{\varphi}$:
$$\brk[a]*{\vec{\varphi}; \vec{d}} = 0 \Longleftrightarrow \sum_{j=1}^{N}\varphi_{j}d_{j} = 0 \Longleftrightarrow \int_{G}f(y)\brk[s]*{\sum_{j = 1}^{N}b_{j}(y)d_{j}}dy = 0 \Longleftrightarrow \int_{G}f(y)v(y)dy = 0$$

Получили что интегральное уравнение с вырожденным ядром совместно тогда и только тогда, когда $f$ ортогонально каждому решению однородного союзного уравнения.

Сформулируем все полученные и доказанные выше результаты в виде теорем Фредгольма:

\begin{theorem}[Первая теорема Фредгольма]
  Если $D(\lambda) \neq 0$, то интегральное уравнение с вырожденным ядром и союзное к нему однозначно разрешимы при любых правых частях из $C(\overline{G})$.
\end{theorem}
\begin{theorem}[Вторая теорема Фредгольма]
   Если $D(\lambda) = 0$, то интегральное уравнение с вырожденным ядром и союзное к нему имеют одинаковое число линейно независимых решений $m = N - \mathrm{rk}\,(E - \lambda A)$
\end{theorem}
\begin{theorem}[Третья теорема Фредгольма]
  Если $D(\lambda) = 0$, то для разрешения интегрального уравнения с вырожденным ядром необходимо и достаточно, чтобы свободный член $f(x) \in C(\overline{G})$ был ортогонален всем решениям союзного уравнения.
\end{theorem}
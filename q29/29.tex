\section{Объёмный ньютонов потенциал и его свойства: гладкость, убывание на бесконечности, результат действия оператора Лапласа на объемный потенциал.}

\begin{itemize}
	\item Функция $E(x) = -\frac{1}{4\pi|x|}$ является решением в обобщенных функциях уравнения $\Delta E(x) = \delta (x)$. Эту запись нужно понимать следующим образом:
    $$
    \int\limits_{\R^3}\brk*{-\frac{1}{4\pi|y|}}\Delta_y \varphi (y) \mathrm{d}y = \varphi(0),\; \varphi \in \mathcal{D}(\R^n)
    $$
\end{itemize}

\begin{definition}
	Функция $V (x)$ вида $V (x) = \int\limits_{\R^3}\frac{\rho (y)}{|x-y|}\mathrm{d}y$ называется \underline{\it \text{объемным ньютоновым потенциалом}}.
\end{definition}

\begin{remark} 
Это свёртка фундаментального решения с функцией $-4\pi\rho (x)$
\end{remark}

\begin{theorem}

	\begin{enumerate}
		\item Пусть $\rho (x)$ \text{--} кусочно-непрерывная, ограниченная, финитная. Тогда $V (x) \in C^1(\R^3)$ и $  V (x) = O\brk*{\frac{1}{|x|}}$ при $x\rightarrow \infty $.
        \item Если $\exists$ область $\Omega \subset \R^3: \;\rho (x) \in C^1(\Omega)$, то $V (x) \in C^2 (\Omega)$ и $\Delta V (x) = -4\pi \rho (x), \;x \in \Omega$.
	\end{enumerate}
\end{theorem}

\begin{proof}

Доказательство проведем в менее общей постановке: считаем $\rho \in C^\infty$ и $\supp{\rho}$ компактом.

	\begin{itemize}
		\item $\exists C: |\rho (x)|\leq C \Forall  x \in \R^3$. Считаем, что $\supp{\rho}  \subset B_A(0)$. Берем $x: |x| > 2A$. Тогда если $|y| \leq A$, то $|y| < \frac{|x|}{2}$.
        \item Оценка:
        	\begin{multline*}
				\ |V (x)| = \Bigl | \int\limits_{\R^3} \frac{\rho (y)}{|x-y|}\mathrm{d}y \Bigr | = {}\\
                {} = \Bigl | \int\limits_{|y|<A} \frac{\rho (y)}{|x-y|}\mathrm{d}y \Bigr | \;\leq\; \int\limits_{|y|<A} \frac{|\rho (y)|}{|x-y|}\mathrm{d}y \;\leq\; \frac{C}{\frac{|x|}{2}}\int\limits_{|y|<A}\frac{\mathrm{d}y}{1} = \frac{8}{3}\frac{\pi C A^3}{|x|} \;\Rightarrow {}\\
                {} \Rightarrow V (x) = O \brk*{\frac{1}{|x|}}.
			\end{multline*}
        \item Пользуемся утверждением из анализа: {\it Пусть $F(x,y)$ и $\frac{\partial F}{\partial x_j}(x,y), \; j= \overline{1,n}$ непрерывны на $\Omega \times G$, где $\Omega \subset  \R^n, G \subset \R^m$. Пусть $g(x)$ абсолютно интегрируема: $\int\limits_G |g(x)| \mathrm{d}x < \infty$. Тогда $\int\limits_G F(x,y)g(y) \mathrm{d}y \in C^1(\overline{\Omega})$ и
        $$
        \frac{\partial}{\partial x_j}\int\limits_G F(x,y)g(y)\mathrm{d}y = \int\limits_G \frac{\partial F}{\partial x_j}(x,y)g(y)\mathrm{d}y. 
        $$}
        \item Пусть $\rho (x) \in C^\infty (\R^3)$ и $\exists A: \rho (x) \equiv 0 \Forall x: |x| > A$. Пусть 
        $$\Omega = \brk[c]*{x: |x| < R};\quad F(x,y) = \rho (x+y)$$
        $$G = \brk[c]*{y: |y| < R+A};\quad g(y) = \frac{1}{|y|}.$$
        При $|x| < R,\; |y| > A+R \; \hookrightarrow |x+y| \;\geq\; |y| - |x| \;\geq\; A+R -R = A \;\Rightarrow \; \rho (x+y) \equiv 0. $
        \item 
        $$
        V (x) = \int\limits_{\R^3}\frac{\rho (x+y)}{|y|}\mathrm{d} y = \int\limits_{|y| < R+A}\frac{\rho (x+y)}{|y|}\mathrm{d} y \quad \Rightarrow\quad \frac{\partial V}{\partial x_j} = \int\limits_{|y| < R+A}\frac{\partial \rho (x+y)}{\partial x_j}\frac{1}{|y|}\mathrm{d} y
        ;$$ Также для остальных производных.
        \item 
        $$
        \begin{CD}
          \Delta V (x) = \int\limits_{\R^3}\frac{\Delta_x \rho (x+y)}{|y|}\mathrm{d}y = -4\pi \int\limits_{\R^3}\brk*{-\frac{1}{4\pi |y|}} @. \underbrace{\Delta_y \rho (x+y)} @.\mathrm{d}y = -4\pi \brk[a]*{\delta(y),\; \rho (x+y)} = -4\pi \rho (x) \\
          @.   @|   @.\\
          @.   {\Delta_x \rho (x+y)}   @.
        \end{CD}
        $$
	\end{itemize}

\end{proof}
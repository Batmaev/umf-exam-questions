\documentclass[../main.tex]{subfiles}
\begin{document}


\section[Неоднородное волновое уравнение в \texorpdfstring{$\R^3$}{R\textasciicircum 3}]{Метод Дюамеля решения задачи Коши для неоднородного волнового уравнения в $\R^3$. Общая формула Кирхгофа.}
% Иваныч техал
\subsection*{Формулировка задачи}

\begin{equation} \label{8_1}
    \begin{cases}
        u_{tt} - a^2\Delta u = f(t, x), & t > 0, x \in \R^3 \\
        \eval{u}_{t=0} = 0, & \\
        \eval{u_t}_{t=0} = 0.
    \end{cases} \;\;\text{ Считаем, что } f(t, x) \in C^{0,2}_{t,x}\, \{t \geq 0,\, x \in \R^3\}
\end{equation}

\Subsection{Метод Дюамеля}

Сведем задачу к задаче Коши для однородного волнового уравнения. Рассмотрим однопараметрическое семейство задач:
\begin{equation} \label{8_2}
\begin{cases}
    w_{tt}(t, x, \tau) - a^2\Delta_x w(t, x, \tau) = 0, & t > \tau;\ x \in \R^3 \\
    \eval{w}_{t=\tau} = 0, & \\
    \eval{w_t}_{t=\tau} = f(\tau, x), &
\end{cases} \;\;\; \tau \geq 0.
\end{equation}

Решение задач семейства получаем по формуле Пуассона-Кирхгофа:

\begin{equation*}
    w(t, x, \tau) = \frac{1}{4\pi a^2(t - \tau)}\oiint\limits_{\abs*{\xi - x}=a(t-\tau)}f(\tau, \xi)\,dS_\xi \quad \in \; C^{2,2,0}_{t,x,\tau}\,\{t - \tau \geq 0,\, x \in \R^3\}
\end{equation*}

(Под $C^{2,2,0}_{t,x,\tau}$ подразумевается, что $D^\alpha_{t, x}\: w(t, x, \tau)\in C \Forall \alpha\colon |\alpha| \leq 2$)

\begin{statement}
    $$
    u(t, x) = \int_0^t w(t, x, \tau)d\tau \text{ -- классическое решение исходной задачи.}
    $$
\end{statement}
\begin{proof}\hfill
\begin{itemize}
    \item $u(t,x) \in C^2\, \{t \geq 0, \; x \in \R^3\}$

    \item $\eval{u}_{t=0} = 0$
    
    \item Вычислим $\eval{u_t}_{t=0}$
    
    Интегралы, в которых и пределы, и подынтегральная функция зависят от параметра, дифференциируются по формуле
    $$\pd{}{t} \int_{a(t)}^{b(t)} f(t,x)\,dx = f(t,b(t)) \cdot b' - f(t,a(t)) \cdot a' + \int_{a(t)}^{b(t)} f'_t(t,x)\,dx$$
    В нашем случае $a = 0,\ b = t,\ b' = 1$. Поэтому
    $$\eval{u_t}_{t=0} = 
    \eval*{\brk*{\underbrace{w(t, x, \tau)|_{\tau = t}}_\text{= 0 из нач. усл. в \eqref{8_2}} + \int_0^t \pd{w}{t}d\tau}}_{t=0} = \int_0^0 \pd{w}{t}\, d\tau = 0$$

    \item $\displaystyle \ \Delta_xu = \int_0^t\Delta_x w(t, x, \tau)d\tau$

    \begin{align*}
        u_{tt} &= \pd{}{t}\int_0^t \pd{w}{t}\, d\tau = w_t(t, x, \tau)|_{\tau = t} + \int_0^t w_{tt} (t, x, \tau)\, d\tau
        = f(\tau, x)|_{\tau = t} + a^2 \int_0^t \Delta_x w(t, x, \tau) \, d\tau \\[0.5em]
        &= f(t, x) + a^2\Delta_x u
    \end{align*}
\end{itemize}

Значит, рассматриваемая функция удовлетворяет уравнению.
\end{proof}
Мы доказали следующую теорему:
\begin{theorem}
    Пусть в  задаче \eqref{8_1} функция $f(t,x)$ такова, что $D^\alpha_x\, f \in C\,\{t \geq 0,\; x \in \R^3\}$\\ 
    Тогда функция 
    \begin{equation} \label{8_3}
        u(t, x) = \int\limits_0^t{ \frac{1}{4\pi a^2(t - \tau)}
        \brk[s]*{\oiint\limits_{\;\abs*{\xi - x} = a(t - \tau)}
        f(\tau, \xi)\, dS_\xi} d\tau}
    \end{equation}
    является классическим решением, причем 
    $$
    D^\alpha_{t, x}\, u \in C\,\{t \geq 0, x \in \R^3\}.
    $$
\end{theorem}
Суть метода Дюамеля: $f(t, x)$ -- это начальные данные в каждый момент времени.

\Subsection{Запаздывающий потенциал}
Преобразуем формулу \eqref{8_3}: 
\begin{align*}
    &\int\limits_0^t \frac{d\tau}{4\pi a^2(t - \tau)}
    \oiint\limits_{\abs*{\xi - x} = a(t - \tau)} f(\tau, \xi)\, dS_\xi = 
    \begin{bmatrix} 
        \rho = a(t - \tau) \\
        \tau = t -\rho/a \\
        d\tau = -d\rho/a 
    \end{bmatrix} = 
    -\int\limits_{at}^0 \frac{1}{4\pi a \rho} \frac{d\rho}{a} 
    \oiint\limits_{\abs*{\xi - x} = \rho} f \brk*{t - \frac{\rho}{a},\; \xi} dS_\xi = \\[1em]
    &= \frac{1}{4\pi a^2} \int\limits_0^{at} d\rho \oiint\limits_{\abs*{\xi - x} = \rho} \frac{f(t - \frac{\abs*{\xi - x}}{a},\; \xi)}{\abs*{\xi - x}}\, dS_\xi
    = \boxed{\frac{1}{4\pi a^2} \iiint\limits_{\abs*{\xi - x} < at}\frac{f(t - \frac{\abs*{\xi - x}}{a}, \xi)}{\abs*{\xi - x}}\;d\xi}
\end{align*}

Выражение в рамке называется \emph{запаздывающим потенциалом}.

\Subsection[Общая формула Кирхгофа]{Общая задача Коши:}
\vspace{-0.5em}
$$
\begin{cases}
    u_{tt} - a^2\Delta_x u = f(t, x), \\
    \eval{u}_{t=0} = u_0(x), \\
    \eval{u_t}_{t=0} = u_1(x).
\end{cases}
$$
\vspace{0em} % По какой-то причине это добавляет ненулевое vspace именно той величины, которое нужно

\begin{theorem}[Формула Кирхгофа]
    Пусть в общей задаче Коши
    $$
    u_0 \in C^3(\R^3), \qquad 
    u_1 \in C^2(\R^3), \qquad 
    f \in C^{0,2}_{t,x}\,\{t \geq 0,\; x \in \R^3\}. $$
    Тогда выражение
    \begin{equation*} \label{8_4}
        u(t, x) = \pd{}{t} \left[ \frac{1}{4\pi a^2t}\oiint\limits_{\abs*{\xi - x}=at}u_0(\xi)\, dS_\xi \right]
        + \frac{1}{4\pi a^2t} \oiint\limits_{\abs*{\xi - x}=at}u_1(\xi)\, dS_\xi 
        + \frac{1}{4\pi a^2} \iiint\limits_{\abs*{\xi - x} < at} \frac{f(t - \frac{\abs*{\xi - x}}{a}, \xi)}{\abs*{\xi - x}}\, d\xi
    \end{equation*}

    является классическим решением общей задачи Коши, \; $u(t,x) \in C^2\,\{t \geq 0,\ x \in \R^3\}$.
\end{theorem}

\end{document}

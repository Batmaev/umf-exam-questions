\section{Сведение задач Дирихле и Неймана для уравнения Лапласа посредством потенциалов к интегральным уравнениям Фредгольма второго рода на границе. Существование и единственность решения внутренней задачи Дирихле и внешней задачи Неймана.}

\Subsection{Внешняя задача Неймана для уравнения Лапласа}

Найти функцию $u(x) \in C^2(\R^3 \backslash \overline{\Omega}) \, \cap \, C(\R^3 \backslash \Omega)$ имеющую ПНП по направлению внешней нормали, и такую, что 

\begin{equation}
 \begin{cases}
 \Delta u(x) = 0, &x \in \R^3 \backslash \overline{\Omega},
  \\
   \big(\frac{\partial u}{\partial \vec{n}}\big)_- \big|_{{\Gamma}} = u_1(x), & x \in {\Gamma}, \quad u_1(x) \in C({\Gamma}),
   \\
   u(x) \to_{|x| \to \infty} 0.\\
 \end{cases}
\end{equation}
Решение этой задачи ищем в виде потенциала простого слоя: 

$u(x) = \int_{\Gamma} \frac{\mu(y)}{|x - y|} dS_y$. Свойства $\Delta u(x) = 0$ и $\lim\limits_{x \to \infty} u(x) \to 0$ выполнены.

Кроме того, $u(x) \in C^2(\R^3 \backslash \overline{\Omega}) \, \cap \, C(\R^3 \backslash \Omega)$ и имеет ПНП. Нужно проверить $\big(\frac{\partial u}{\partial \vec{n}}\big)_- \big|_{{\Gamma}} = u_1(x), \, x \in {\Gamma}$. 

$\bullet$ Пусть $$x^0 \in {\Gamma} \Rightarrow \bigg(\frac{\partial u}{\partial \vec{n}}\bigg)_-(x^0) = -2\pi \mu(x^0) + \int_{\Gamma} \frac{(y - x^0, \vec{n}(x^0))}{|x^0 - y|^3}\mu(y)dS_y = u_1(x^0)$$

$\mu(x^0) = \frac{1}{2\pi}\int_{\Gamma} \frac{(y - x^0, \vec{n}(x^0))}{|x^0 - y|^3}\mu(y)dS_y - \frac{1}{2\pi}u_1(x^0), \, x^0 \in {\Gamma} $ -- интегральное уравнение Фредгольма 2-го рода с интегральным оператором и полярным ядром.

Уравнение однозначно разрешимо $\Leftrightarrow$ уравнение $\mu_*(x^0) = \frac{1}{2\pi}\int_{\Gamma} \frac{(y - x^0, \vec{n}(x^0))}{|x^0 - y|^3}\mu(y)dS_y \equiv 0 $ имеет только тривиальное решение. 

Для этого покажем, что $V^{(0)}(x) = \int_{\Gamma} \frac{\mu(y)}{|x - y|}dS_y \equiv 0$

Ясно, что уравнение на $\mu_*$ получается таким же образом как и уравнение на $\mu$, но из задачи Неймана 
\begin{equation}
 \begin{cases}
 \Delta u(x) = 0, &x \in \R^3 \backslash \overline{\Omega},
  \\
   \big(\frac{\partial u}{\partial \vec{n}}\big)_- \big|_{{\Gamma}} = u_1(x), & x \in {\Gamma},
   \\
   u(x) \to_{|x| \to \infty} 0. \\
 \end{cases}
\end{equation}
В силу единственности решение этой задачи $V^{(0)} = 0 \, (x \in \R^3 \backslash \overline{\Omega})$. Но $V^{(0)} \in C(\R^3) \Rightarrow V^{(0)} \equiv 0$ на $x \in \R^3 \backslash \Omega$. Функция $V^{(0)}$ удовлетворяет внутренней задаче Дирихле

\begin{equation}
 \begin{cases}
 \Delta V^{(0)} = 0, & x \in \Omega
  \\
  V^{(0)}(x) = 0, &x \in {\Gamma};
 \end{cases}
\end{equation}

$\Rightarrow V^{(0)} \equiv 0 \Forall x \in \Omega$.

Итак, $V^{(0)}(x) \equiv 0 \Forall x \in \R^3$. 

Но $\big(\frac{\partial V^{(0)}}{\partial \vec{n}}\big) \big|_+(x^0) - \big(\frac{\partial V^{(0)}}{\partial \vec{n}}\big) \big|_-(x^0) = 4\pi \mu_*(x^0), \, x^0 \in {\Gamma} \Rightarrow \mu_*(x^0) \equiv 0 \Forall x^0 \in {\Gamma}$.

Итак, по теореме Фредгольма об альтернативе уравнение 

(*) $\mu(x^0) = \frac{1}{2\pi}\int_{\Gamma} \frac{(y - x^0, \vec{n}(x^0))}{|x^0 - y|^3}\mu(y)dS_y - \frac{1}{2\pi}u_1(x^0), \, x \in {\Gamma}$ однозначно разрешимо.

\Subsection{Внутренняя задача Дирихле для уравнения Лапласа}

Найти функцию $u(x) \in C^2(\Omega) \cap C(\overline{\Omega})$ такую, что:

\begin{equation}
  \begin{cases}
	\Delta u(x) = 0, & x \in \Omega, \\
	\eval{u}_{\Gamma} = u_0(x), & x \in {\Gamma}, \quad u_0(x) \in C({\Gamma}).\\
\end{cases}
\end{equation}

Решение этой задачи ищем в виде потенциала двойного слоя 

$u(x) = \int_{\Gamma} \frac{(x - y, \vec{n}(y))}{|x - y|^3}\nu(y)dS_y, \, \nu(x) \in C({\Gamma})$

Условия $\Delta u = 0 \in \Omega$ и $u(x) \in C^2(\Omega) \cap C(\overline{\Omega})$ уже выполнены. Осталось проверить $\eval{u}_{{\Gamma}} = u_0(x), \, x \in {\Gamma}$.

$\bullet$ Для потенциала двойного слоя справедлива формула скачка: $u_+(x^0) = u(x^0) - 2\pi\nu(x^0) \Rightarrow u_0(x^0) = \int_{\Gamma} \frac{(x^0 - y, \vec{n}(y))}{|x^0 - y|^3}\nu(y)dS_y - 2\pi\nu(x^0)$.

$\Rightarrow \nu(x^0) = \frac{1}{2\pi} \int_{\Gamma} \frac{(x^0 - y, \vec{n}(y))}{|x^0 - y|^3}\nu(y)dS_y - \frac{1}{2\pi} u_0(x^0), \, x^0 \in {\Gamma}$. Ядро, транспонированное к тому, что стоит в (*).

Т.к. (*) однозначно разрешмимо, это уравнение тоже имеет единственное решение.

\begin{theorem} Пусть $\Omega$ -- ограниченная область в $\R^3$ с границей ${\Gamma} \in C^2$. Тогда у внутренней задачи Дирихле $\Forall u_1(x) \in C({\Gamma})$, а также у внешней задачи Неймана $\Forall u_0(x) \in C({\Gamma})$ существует единственное классическое решение.
\end{theorem} 


\section{Задача Дирихле для уравнения Пуассона в ограниченной области. Единственность классического решения задачи Дирихле для уравнения Пуассона при непрерывной граничной функции (в классе $C^2(\Omega) \cap C(\overline{\Omega})$).}
%Никитушка отмочил красоту

$\Omega$ -- ограниченная область 

\begin{equation}
  \begin{cases}\label{equatation23.1}
  \Delta u(x) = f(x), & x \in \Omega,\\ 
  \eval{u(x)}_{\partial \Omega} = u_0(x), & x \in \partial \Omega.\\
  \end{cases}
\end{equation} 
  
 \begin{definition}
 Классическое решение задачи Дирихле -- функция
  $u(x) \in C^2(\Omega) \cap C(\overline{\Omega})$,
  удовлетворяющая уравнению и граничному условию. (Раньше было $u(x) \in C^2(\Omega) \cap C^{\boxed{1}}(\overline{\Omega})$, единица была нужна для формул Грина, теперь убираем ее).
 \end{definition}
 \Subsection{Теорема единственности}
 \begin{theorem}[Единственности] Не может существовать более 1 классического решения задачи Дирихле (\ref{equatation23.1}).
 
 \begin{proof}
 Пусть $u_1$ и $u_2$ -- классические решения (\ref{equatation23.1}). Тогда $v(x) = u_1 - u_2$ -- классическое решение полностью однородной задачи:
 
 \begin{equation}
  \begin{cases}\label{equatation23.2}
  \Delta v(x) \equiv 0, & x \in \Omega,\\ 
  \eval{v(x)}_{\partial \Omega} = 0, & x \in \partial \Omega.\\
  \end{cases}
\end{equation}
 
 Согласно принципу максимума, $\abs*{v(x)} \leq \max\limits_{x \in \partial\Omega} \abs*{v(x)} = 0 \Rightarrow v(x) \equiv 0$.
 \end{proof}
 \end{theorem}
